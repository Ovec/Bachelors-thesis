% ============================================================================ %
%
%           Šablona bakalářské/diplomové práce
%
% Autor:    Ing. Jozef Říha (2006-05-04), od té doby šablonu udržuje
%           Ing. Pavel Tomášek, Ph.D. (tomasek@utb.cz)
%
% Verze:    2021-05-04
%
% Kódování: UTF-8 (kontrolní řetězec: žluťoučký kůň úpěl ďábelšké ódy)
%
% Sazba:    pdflatex prace.tex && pdflatex prace.tex
%           (nutné dvakrát pro korektní vložení citací a jiných referencí),
%           v případě umístění literatury do externího bib souboru je třeba volat
%           pdflatex prace.tex && bibtex prace && pdflatex prace.tex && pdflatex prace.tex
%
% Tip:      Ve správně vysázeném českém textu by na konci řádku neměla zůstant
%           samotná jednopísmenná předložka či spojka. Na takové místo se vkládá
%           nezalomitelná mezera pomocí symbolu ~. Existuje program, který umí
%           zpracovat celý TeX dokument najednou podle českých konvencí:
%           http://petr.olsak.net/ftp/olsak/vlna/
%
% Pozor:    Vzhledem k požadovanému standardu PDF/A nesmí vložené obrázky 
%           obsahovat alfa kanál (průhlednost).
%
% ============================================================================ %


\documentclass[a4paper,12pt]{article}

% Definice vzhledu a nastavení se načítá z následujícího souboru (netřeba editovat)
\input{tex/UTB.tex}

% Uživatelské definice -- upravte dle požadavků
\nastavfakultu{FAI}
	% FAI  -- pro Fakultu aplikované informatiky
	% FAME -- pro Fakultu managementu a ekonomiky
	% FHS  -- pro Fakultu humanitních studií
	% FLKR -- pro Fakultu logistiky a krizového řízení
	% FMK  -- pro Fakutlu mutimediálních komunikací
	% FT   -- pro Fakultu technologickou
	% UNI  -- pro Univerzitní institut
\nastavtyp{BP}
	% BP   -- bakalářská práce
	% DP   -- diplomová práce
\nastavrok{2023}
	% zadejte rok místo "xxxx"
\nastavjazyk{EN}
	% CZ   -- práce bude v českém jazyce
	% EN   -- práce bude v anglickém jazyce

% Lze přidat vertikalni odsazeni nad (prvni parametr) a pod (druhy parametr)
% obrázky, tabulky i rovnice/soustavy rovnic
\nastavmezerukolemobrazku{0mm}{0mm} 
\nastavmezerukolemtabulek{0mm}{0mm}
\nastavmezerukolemrovnic{0mm}{0mm}

\nastavautora{Miroslav Šiřina}
\nastavnazevcz{Název práce česky (max. 2 řádky)}
\nastavnazeven{Postgres Lifecycle Management Operators in Kubernetes} % Jen u anglicky psané práce
\nastavabstraktcz{Tato bakalářská práce je zaměřena na poskytnutí doporučení zainteresovaným stranám v rámci výběru vhodného operátora pro správu životního cyklu systému Postgres v prostředí Kubernetes. Práce se zabývá životním cyklem Postgres, dále relevantními metrikami pro testování a také samotným testováním operátorů a to: Crunchy Postgres for Kubernetes, CloudNativePG, StackGres Operator a Percona Operator for PostgreSQL. Pokud jde o výkon a spolehlivost, byl Crunchy Postgres for Kubernetes doporučen jako nejvhodnější operátor. StackGres Operator byl naopak vyhodnocen jako nejlepší z hlediska jednoduchosti použití a udržovanosti. Tato práce navrhuje další studie na téma bezpečnosti operátorů životního cyklu systému Postgres v prostředí Kubernetes.}
\nastavabstrakten{This bachelor's thesis is aimed at providing recommendations to stakeholders in selecting a suitable operator for Postgres lifecycle management in a Kubernetes environment. The thesis covers the Postgres lifecycle, relevant metrics for testing, and also the testing of the operators themselves namely; Crunchy Postgres for Kubernetes, CloudNativePG, StackGres Operator, and Percona Operator for PostgreSQL. In terms of performance and reliability, Crunchy Postgres for Kubernetes was recommended as the most suitable operator. StackGres Operator, on the other hand, was rated as the best in terms of ease of use and maintenance. This thesis proposes further studies on the security of Postgres lifecycle operators in a Kubernetes environment.}
\nastavklicovaslovacz{Přehled klíčových slov}
\nastavklicovaslovaen{Some keywords}

% Následující příkaz nastaví standard PDF/A-1b
\aplikujpdfa

% ============================================================================ %
\begin{document}

\titulnistrana

\zadani

\prohlaseni

\abstraktaklicovaslova


% ============================================================================ %
\clearpage
\thispagestyle{empty}
Zde je místo pro případné poděkování, motto, úryvky knih, básní atp.


% ============================================================================ %
\obsah  % Obsah je generován automaticky


% ============================================================================ %
\OdsazovaniOdstavcuStart % Nastaví odsazování odstavců dle zvoleného jazyka

% ============================================================================ %
% Encoding: UTF-8 (žluťoučký kůň úpěl ďábelšké ódy)
% ============================================================================ %

% ============================================================================ %
\nn{Introduction}
TBD - První řádek prvního odstavce v kapitole či podkapitole se neodsazuje, ostatní ano. Vertikální odsazení mezy odstavci je typycké pro anglickou sazbu; czech babel toto respektuje, netřeba do textu přidávat jakékoliv explicitní formátování, viz ukázka sazby tohoto textu s následujícím odstavcem).

Formátování druhého odstavce. Text text text text text text text text text text text text.


% ============================================================================ %
\cast{Theory}

\n{1}{Postgres}
\n{2}{History}
PostgreSQL is a powerful object-relational database management system (ORDBMS) derived from the POSTGRES package written at the University of California at Berkeley. The first version of POSTGRES was released in June 1989. POSTGRES has been used in many applications, including financial data analysis systems, asteroid tracking databases, medical information database, and several geographic information systems. The size of external community users has nearly doubled by 1993.\cite{the_postgresql_global_development_group_postgresql_2022}
    
 POSTGRES was using its POSTQUEL query language from version one until 1995, when Andrew Yu and Jolly Chen introduced SQL to POSTGRES. The name has changed to Postgres95. Postgres95 was completely ANSI C code reduced by 25 \% and was 30 – 50 \% faster than Postgres 4.2.  \cite{the_postgresql_global_development_group_postgresql_2022}

It was clear by 1996 that the name would not stand the test of time therefore it has been renamed to PostgreSQL. As stated by PostgreSQL documentation \cite{the_postgresql_global_development_group_postgresql_2022}: “Many people continue to refer to PostgreSQL as “Postgres” (now rarely in all capital letters) because of tradition or because it is easier to pronounce. This usage is widely accepted as a nickname or alias.“ This thesis will use Postgres as an alias for PostgreSQL as well. 
\cite{the_postgresql_global_development_group_postgresql_2022}

\n{2}{Current State of Postgres}
More than 30 years after the first version Postgres was considered the most used ORDBMS for professional developers in Stack Overflow survey \cite{so2022survey}. According to Riggs \cite{Riggs2022}: “The PostgreSQL feature set attracts serious users who have serious applications. Financial services companies may be PostgreSQL's largest user group, although governments, telecommunication companies, and many other segments are strong users as well.“ It is fully ACID compliant \cite{juba2015learning} and supports many kinds of data models such as relational, document, and key/value. \cite{Riggs2022}

%Database engines survey https://db-engines.com/en/ranking
%Describe postgres usage
%Describe scaling etc.
%DSK time consuming Postgres

\n{1}{Kubernetes}
Kubernetes, also known as K8s, is an open-source platform for automating deployment, scaling, and management of containerized applications. It provides a way to manage and orchestrate containers, which are units of software that package up an application and its dependencies into a single, isolated package that can run consistently on any infrastructure. \cite{vayghan2019kubernetes}

Kubernetes provides several key features, including:
\begin{itemize}
\item \textbf{Service discovery:} bla blab olkjdlkaj lkjasdkla sjd
\item \textbf{Load balancing:}
\item \textbf{Storage Orchestration:}
\item \textbf{Automated rollouts and rollbacks:}
\item \textbf{Automatic bin packing:}
\item \textbf{Self healing:}
\item \textbf{Secret and configuration management:}
\end{itemize}


\n{2}{Microservices}
\n{2}{Architecture}



%\obr{2022 Developer Survey \cite{so2022survey}}{}{1}{graphics/postgres_stack_overflow_survey.png}





\n{1}{Nadpisy a podnadpisy}
Na této stránce je k vidění způsob tvorby různých úrovní nadpisů.

\n{2}{Podnadpis A}
Text

\n{2}{Podnadpis B}
Text

\n{2}{Podnadpis C}
Text

\n{3}{Podpodnadpis alfa}
Text

\n{3}{Podpodnadpis beta}
Text

\n{3}{Podpodnadpis gama}
Text

\n{2}{Podnadpis D}
Text


\n{1}{Vkládání obrázků, tabulek a citací}
Níže následují ukázky vložení obrázku, tabulky a různorodých citací.


\n{2}{Obrázek}
Obrázek \ref{fig:logo} prezentuje logo Fakulty aplikované informatiky.

% Obrázek lze vkládat pomocí následujícího zjednodušeného stylu, nebo klasickým LaTex způsobem
% Pozor! Obrázek nesmí obsahovat alfa kanál (průhlednost). Jde to totiž proti požadovanému standardu PDF/A.
\obr{Popisek obrázku}{fig:logo}{0.5}{graphics/logo/fai_logo_cz.png}


\n{2}{Tabulka}
Tabulka \ref{tab:priklad} obsahuje dva řádky a celkem 7 sloupců.

% Tabulku lze vkládat pomocí následujícího zjednodušeného stylu, nebo klasickým LaTex způsobem
\tab{Popisek tabulky}{tab:priklad}{0.65}{|l|c|c|c|c|c|r|}{
  \hline
   & 1 & 2 & 3 & 4 & 5 & Cena [Kč] \\ \hline
  \emph{F} & (jedna) & (dva) & (tři) & (čtyři) & (pět) & 300 \\ \hline
}


\n{2}{Citování}
Následuje ukázka odkazování na různé zdroje:
\begin{itemize}
	\item kniha \cite{HRW1997},
	\item kapitola v knize \cite{Delorme2006},
	\item článek v odborném žurnálu \cite{Bourreau2006},
	\item konferenční příspěvek \cite{Judish1999},
	\item doktorská práce \cite{Valente2005},
	\item technická zpráva \cite{Fralick1997},
	\item webová stránka \cite{WWWCST}.
\end{itemize}


% ============================================================================ %

% Pokud Vaše práce obsahuje analytickou část, stačí odkomentovat nasledujících dva řádky
%\cast{Analytická část}
%\n{1}{Nadpis}


% ============================================================================ %
\cast{Praktická část}
\n{1}{Nadpis první kapitoly praktické části}
Text


% ============================================================================ %
\nn{Závěr}
Text závěru.


% ============================================================================ %
 % Hlavni text prace

\OdsazovaniOdstavcuStop


% ============================================================================ %
\seznamlitbib


% ============================================================================ %
% ============================================================================ %
% Encoding: UTF-8 (žluťoučký kůň úpěl ďábelšké ódy)
% ============================================================================ %

\seznamzkr

\begin{tabular}{ll}
    ORDBMS & Object-relational Database Management System  \\
    ACID   & Atomicity, Consistency, Isolation, Durability \\
    K8s    & Kubernetes                                    \\
    PGO    & Crunchy Postgres for Kubernetes               \\
    EDBO   & EDB Postgres for Kubernetes Operator          \\
    CPU    & Central Processing Unit                       \\
    PTFE   & Polytetrafluoroethylene                       \\
    VNA    & Vector Network Analyser                       \\
\end{tabular}

% ============================================================================ %
 % Seznam zkratek


% ============================================================================ %
\seznamobr  % Seznam je generován automaticky


% ============================================================================ %
\seznamtab  % Seznam je generován automaticky


% ============================================================================ %
% ============================================================================ %
% Encoding: UTF-8 (žluťoučký kůň úpěl ďábelšké ódy)
% ============================================================================ %

\listofappendices

% \priloha{Název přílohy}
% Obsah přílohy

\priloha{General test plan}
\label{chap:generalTestPlan}

\begin{itemize}
    \item Test plan ID: TP0
    \item Context of testing:
          \begin{itemize}
              \item Project: Bachelor’s thesis.
              \item Test levels: Acceptance testing.
              \item Test types: Static and dynamic.
              \item Test items:
                    \begin{itemize}
                        \item Crunchy Postgres for Kubernetes Operator v5.3.1.
                        \item CloudNativePG Operator v1.20.0.
                        \item StackGres Operator v1.4.3.
                        \item Percona Operator for Postgres 1.4.0.
                    \end{itemize}
              \item Test scope: Operator, Operator’s documentation, Operator’s repository.
              \item Test basis: Defined criteria.
          \end{itemize}
    \item Risk register:
          \begin{itemize}
              \item Limited staff and time might prevent thorough testing of all features and functionalities of the software during acceptance testing.
              \item Inadequately trained staff might struggle to design effective test cases, which could result in missed defects and lower overall testing effectiveness.
              \item Due to the lack of expertise among staff members, the software's readiness for production might be inaccurately assessed, leading to incorrect conclusions about its quality and suitability for release
          \end{itemize}
    \item Test strategy:
          \begin{itemize}
              \item General: The purpose of testing is to evaluate the ability of Operators to fulfill the desired criterias, and to provide information for making informed decisions on which Operator to select in last chapter. Non-functional requirements will be tested with static and dynamic test techniques.
              \item Test levels: Acceptance testing
              \item Test deliverables: Test plan, test model specification, test procedure specification, incident reports, test status reports, test competition reports.
              \item Test design techniques: Exploratory Testing, Use cases, Walkthroughs
              \item Entry criteria: Created environments.
              \item Exit criteria: Decision metrics were collected.
              \item Test competition criteria: All criteria covered by at least one test case.
              \item Degree of independence: No connection between tested Operators and tester. Tester is fully independent.
              \item Metrics to be collected:
                    \begin{itemize}
                        \item Static testing: Vulnerability analysis (number of vulnerabilities and their severity), Repository review (sum of issues, sum of repaired issues, sum of stars, sum of commits), Documentation review (examples, training needed),
                        \item Dynamic testing:  Sum of passed, ignored and failed tests. Peromance described by transactions per second.
                    \end{itemize}
              \item Test data requirements:
                    \begin{itemize}
                        \item Crunchy Postgres for Kubernetes Operator v5.3.1
                              \begin{itemize}
                                  \item PGO: https://github.com/CrunchyData/postgres-operator-examples
                                  \item PGODOC: https://access.crunchydata.com/documentation/postgres-operator/v5/
                                  \item PGOREPO: https://github.com/CrunchyData/postgres-operator
                              \end{itemize}

                        \item CloudNativePG Operator version 1.20.0
                              \begin{itemize}
                                  \item CNPGO: https://raw.githubusercontent.com/cloudnative-pg/cloudnative-pg/release-1.20/releases/cnpg-1.20.0.yaml
                                  \item CNPGODOC: https://cloudnative-pg.io/documentation/1.20/
                                  \item CNPGOREPO: https://github.com/cloudnative-pg/cloudnative-pg
                              \end{itemize}

                        \item StackGres Operator version 1.4.3
                              \begin{itemize}
                                  \item SPGO: https://stackgres.io/downloads/stackgres-k8s/stackgres/helm/
                                  \item SPGODOC: https://stackgres.io/doc/1.4/
                                  \item SPGOREPO: https://gitlab.com/ongresinc/stackgres
                              \end{itemize}
                        \item Percona Operator for PostgreSQL version 1.4.0
                              \begin{itemize}
                                  \item PPOO: https://raw.githubusercontent.com/percona/percona-postgresql-operator/v1.4.0/deploy/operator.yaml
                                  \item PPODOC: https://docs.percona.com/percona-operator-for-postgresql/index.html
                                  \item PPOREPO: https://github.com/percona/percona-postgresql-operator
                              \end{itemize}
                    \end{itemize}
              \item Test environment requirements:
                    \begin{itemize}
                        \item Kind Kubernetes cluster with two worker nodes for all dynamic test except performance tests, installed on Unix/Linux compatible machine.
                        \item Google Kubernetes Engine with two worker nodes.
                        \item Terraform
                        \item Trivy security scanner.
                        \item Kubectl kubernetes controll tool.
                        \item EXCEL.
                    \end{itemize}
              \item Retesting: Retesting is not needed.
              \item Reggresion testing: Reggresion testing is not needed.
              \item Testing activities and estimates:
                    \begin{itemize}
                        \item Environment setup – 30m.
                        \item Repository walkthrough – 2h/Operator.
                        \item Documentation walkthrough – 2h/Operator.
                        \item Deployment and configuration – 4h/Operator.
                        \item Performance – 4h/Operator.
                        \item Operabilitiy and documentation – 8h/Operator.
                        \item Test completion report – 1h/testing day.
                    \end{itemize}

              \item Staffing (roles and responsibilities)
                    \begin{itemize}
                        \item Roles: Test architect, test manager, test designer, test automator, tester and test analyst
                        \item Staff: Miroslav Šiřina.
                    \end{itemize}
              \item Training needed
                    \begin{itemize}
                        \item Test management.
                        \item Test design.
                        \item Test analyst.
                        \item Trivy and results interpretation skills.
                    \end{itemize}
              \item Test priorities
                    \begin{itemize}
                        \item Static tests have higher priority to dynamic.
                        \item Critical features have higher priority.
                    \end{itemize}
              \item Schedule
                    \begin{itemize}
                        \item May 1st repositories and documentations walkthroughs.
                        \item May 2nd vulnerability analysis.
                        \item May 3rd - 8th operability testing.
                        \item May 9th performance testing.
                        \item May 10th – 11th Testing closure.
                    \end{itemize}
          \end{itemize}
\end{itemize}

% ============================================================================ %
 % Prilohy


% ============================================================================ %

\end{document}

% ============================================================================ %
