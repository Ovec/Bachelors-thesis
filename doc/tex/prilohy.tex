% ============================================================================ %
% Encoding: UTF-8 (žluťoučký kůň úpěl ďábelšké ódy)
% ============================================================================ %

\listofappendices

% \priloha{Název přílohy}
% Obsah přílohy

\priloha{General test plan}
\label{chap:generalTestPlan}

\begin{itemize}
    \item Test plan ID: TP0
    \item Context of testing:
          \begin{itemize}
              \item Project: Bachelor’s thesis.
              \item Test levels: Acceptance testing.
              \item Test types: Static and dynamic.
              \item Test items:
                    \begin{itemize}
                        \item Crunchy Postgres for Kubernetes Operator v5.3.1.
                        \item CloudNativePG Operator v1.20.0.
                        \item StackGres Operator v1.4.3.
                        \item Percona Operator for Postgres 1.4.0.
                    \end{itemize}
              \item Test scope: Operator, Operator’s documentation, Operator’s repository.
              \item Test basis: Defined criteria.
          \end{itemize}
    \item Risk register:
          \begin{itemize}
              \item Limited staff and time might prevent thorough testing of all features and functionalities of the software during acceptance testing.
              \item Inadequately trained staff might struggle to design effective test cases, which could result in missed defects and lower overall testing effectiveness.
              \item Due to the lack of expertise among staff members, the software's readiness for production might be inaccurately assessed, leading to incorrect conclusions about its quality and suitability for release
          \end{itemize}
    \item Test strategy:
          \begin{itemize}
              \item General: The purpose of testing is to evaluate the ability of Operators to fulfill the desired criterias, and to provide information for making informed decisions on which Operator to select in last chapter. Non-functional requirements will be tested with static and dynamic test techniques.
              \item Test levels: Acceptance testing
              \item Test deliverables: Test plan, test model specification, test procedure specification, incident reports, test status reports, test competition reports.
              \item Test design techniques: Exploratory Testing, Use cases, Walkthroughs
              \item Entry criteria: Created environments.
              \item Exit criteria: Decision metrics were collected.
              \item Test competition criteria: All criteria covered by at least one test case.
              \item Degree of independence: No connection between tested Operators and tester. Tester is fully independent.
              \item Metrics to be collected:
                    \begin{itemize}
                        \item Static testing: Vulnerability analysis (number of vulnerabilities and their severity), Repository review (sum of issues, sum of repaired issues, sum of stars, sum of commits), Documentation review (examples, training needed),
                        \item Dynamic testing:  Sum of passed, ignored and failed tests. Peromance described by transactions per second.
                    \end{itemize}
              \item Test data requirements:
                    \begin{itemize}
                        \item Crunchy Postgres for Kubernetes Operator v5.3.1
                              \begin{itemize}
                                  \item PGO: https://github.com/CrunchyData/postgres-operator-examples
                                  \item PGODOC: https://access.crunchydata.com/documentation/postgres-operator/v5/
                                  \item PGOREPO: https://github.com/CrunchyData/postgres-operator
                              \end{itemize}

                        \item CloudNativePG Operator version 1.20.0
                              \begin{itemize}
                                  \item CNPGO: https://raw.githubusercontent.com/cloudnative-pg/cloudnative-pg/release-1.20/releases/cnpg-1.20.0.yaml
                                  \item CNPGODOC: https://cloudnative-pg.io/documentation/1.20/
                                  \item CNPGOREPO: https://github.com/cloudnative-pg/cloudnative-pg
                              \end{itemize}

                        \item StackGres Operator version 1.4.3
                              \begin{itemize}
                                  \item SPGO: https://stackgres.io/downloads/stackgres-k8s/stackgres/helm/
                                  \item SPGODOC: https://stackgres.io/doc/1.4/
                                  \item SPGOREPO: https://gitlab.com/ongresinc/stackgres
                              \end{itemize}
                        \item Percona Operator for PostgreSQL version 1.4.0
                              \begin{itemize}
                                  \item PPOO: https://raw.githubusercontent.com/percona/percona-postgresql-operator/v1.4.0/deploy/operator.yaml
                                  \item PPODOC: https://docs.percona.com/percona-operator-for-postgresql/index.html
                                  \item PPOREPO: https://github.com/percona/percona-postgresql-operator
                              \end{itemize}
                    \end{itemize}
              \item Test environment requirements:
                    \begin{itemize}
                        \item Kind Kubernetes cluster with two worker nodes for all dynamic test except performance tests, installed on Unix/Linux compatible machine.
                        \item Google Kubernetes Engine with two worker nodes.
                        \item Terraform
                        \item Trivy security scanner.
                        \item Kubectl kubernetes controll tool.
                        \item EXCEL.
                    \end{itemize}
              \item Retesting: Retesting is not needed.
              \item Reggresion testing: Reggresion testing is not needed.
              \item Testing activities and estimates:
                    \begin{itemize}
                        \item Environment setup – 30m.
                        \item Repository walkthrough – 2h/Operator.
                        \item Documentation walkthrough – 2h/Operator.
                        \item Deployment and configuration – 4h/Operator.
                        \item Performance – 4h/Operator.
                        \item Operabilitiy and documentation – 8h/Operator.
                        \item Test completion report – 1h/testing day.
                    \end{itemize}

              \item Staffing (roles and responsibilities)
                    \begin{itemize}
                        \item Roles: Test architect, test manager, test designer, test automator, tester and test analyst
                        \item Staff: Miroslav Šiřina.
                    \end{itemize}
              \item Training needed
                    \begin{itemize}
                        \item Test management.
                        \item Test design.
                        \item Test analyst.
                        \item Trivy and results interpretation skills.
                    \end{itemize}
              \item Test priorities
                    \begin{itemize}
                        \item Static tests have higher priority to dynamic.
                        \item Critical features have higher priority.
                    \end{itemize}
              \item Schedule
                    \begin{itemize}
                        \item May 1st repositories and documentations walkthroughs.
                        \item May 2nd vulnerability analysis.
                        \item May 3rd - 8th operability testing.
                        \item May 9th performance testing.
                        \item May 10th – 11th Testing closure.
                    \end{itemize}
          \end{itemize}
\end{itemize}

% ============================================================================ %
