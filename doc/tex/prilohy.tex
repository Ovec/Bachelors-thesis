% ============================================================================ %
% Encoding: UTF-8 (žluťoučký kůň úpěl ďábelšké ódy)
% ============================================================================ %

\listofappendices

% \priloha{Název přílohy}
% Obsah přílohy

\priloha{General test plan}
\label{chap:generalTestPlan}

\begin{itemize}
    \item Test plan ID: TP0
    \item Context of testing:
          \begin{itemize}
              \item Project: Bachelor’s thesis.
              \item Test levels: Acceptance testing.
              \item Test types: Static and dynamic.
              \item Test items:
                    \begin{itemize}
                        \item Crunchy Postgres for Kubernetes Operator v5.3.1.
                        \item CloudNativePG Operator v1.20.0.
                        \item StackGres Operator v1.4.3.
                        \item Percona Operator for Postgres 1.4.0.
                    \end{itemize}
              \item Test scope: Operator, Operator’s documentation, Operator’s repository.
              \item Test basis: Defined criteria.
          \end{itemize}
    \item Risk register:
          \begin{itemize}
              \item Limited staff and time might prevent thorough testing of all features and functionalities of the software during acceptance testing.
              \item Inadequately trained staff might struggle to design effective test cases, which could result in missed defects and lower overall testing effectiveness.
              \item Due to the lack of expertise among staff members, the software's readiness for production might be inaccurately assessed, leading to incorrect conclusions about its quality and suitability for release
          \end{itemize}
    \item Test strategy:
          \begin{itemize}
              \item General: The purpose of testing is to evaluate the ability of Operators to fulfill the desired criterias, and to provide information for making informed decisions on which Operator to select in last chapter. Non-functional requirements will be tested with static and dynamic test techniques.
              \item Test levels: Acceptance testing
              \item Test deliverables: Test plan, test model specification, test procedure specification, incident reports, test status reports, test competition reports.
              \item Test design techniques: Exploratory Testing, Use cases, Walkthroughs
              \item Entry criteria: Created environments.
              \item Exit criteria: Decision metrics were collected.
              \item Test competition criteria: All criteria covered by at least one test case.
              \item Degree of independence: No connection between tested Operators and tester. Tester is fully independent.
              \item Metrics to be collected:
                    \begin{itemize}
                        \item Static testing: Vulnerability analysis (number of vulnerabilities and their severity), Repository review (sum of issues, sum of repaired issues, sum of stars, sum of commits), Documentation review (examples, training needed),
                        \item Dynamic testing:  Sum of passed, ignored and failed tests. Peromance described by transactions per second.
                    \end{itemize}
              \item Test data requirements:
                    \begin{itemize}
                        \item Crunchy Postgres for Kubernetes Operator v5.3.1
                              \begin{itemize}
                                  \item PGO: https://github.com/CrunchyData/postgres-operator-examples
                                  \item PGODOC: https://access.crunchydata.com/documentation/postgres-operator/v5/
                                  \item PGOREPO: https://github.com/CrunchyData/postgres-operator
                              \end{itemize}

                        \item CloudNativePG Operator version 1.20.0
                              \begin{itemize}
                                  \item CNPGO: https://raw.githubusercontent.com/cloudnative-pg/cloudnative-pg/release-1.20/releases/cnpg-1.20.0.yaml
                                  \item CNPGODOC: https://cloudnative-pg.io/documentation/1.20/
                                  \item CNPGOREPO: https://github.com/cloudnative-pg/cloudnative-pg
                              \end{itemize}

                        \item StackGres Operator version 1.4.3
                              \begin{itemize}
                                  \item SPGO: https://stackgres.io/downloads/stackgres-k8s/stackgres/helm/
                                  \item SPGODOC: https://stackgres.io/doc/1.4/
                                  \item SPGOREPO: https://gitlab.com/ongresinc/stackgres
                              \end{itemize}
                        \item Percona Operator for PostgreSQL version 1.4.0
                              \begin{itemize}
                                  \item PPOO: https://raw.githubusercontent.com/percona/percona-postgresql-operator/v1.4.0/deploy/operator.yaml
                                  \item PPODOC: https://docs.percona.com/percona-operator-for-postgresql/index.html
                                  \item PPOREPO: https://github.com/percona/percona-postgresql-operator
                              \end{itemize}
                    \end{itemize}
              \item Test environment requirements:
                    \begin{itemize}
                        \item Kind Kubernetes cluster with two worker nodes for all dynamic test except performance tests, installed on Unix/Linux compatible machine.
                        \item Google Kubernetes Engine with two worker nodes.
                        \item Terraform
                        \item Trivy security scanner.
                        \item Kubectl kubernetes controll tool.
                        \item EXCEL.
                    \end{itemize}
              \item Retesting: Retesting is not needed.
              \item Reggresion testing: Reggresion testing is not needed.
              \item Testing activities and estimates:
                    \begin{itemize}
                        \item Environment setup – 30m.
                        \item Repository walkthrough – 2h/Operator.
                        \item Documentation walkthrough – 2h/Operator.
                        \item Deployment and configuration – 4h/Operator.
                        \item Performance – 4h/Operator.
                        \item Operabilitiy and documentation – 8h/Operator.
                        \item Test completion report – 1h/testing day.
                    \end{itemize}

              \item Staffing (roles and responsibilities)
                    \begin{itemize}
                        \item Roles: Test architect, test manager, test designer, test automator, tester and test analyst
                        \item Staff: Miroslav Šiřina.
                    \end{itemize}
              \item Training needed
                    \begin{itemize}
                        \item Test management.
                        \item Test design.
                        \item Test analyst.
                        \item Trivy and results interpretation skills.
                    \end{itemize}
              \item Test priorities
                    \begin{itemize}
                        \item Static tests have higher priority to dynamic.
                        \item Critical features have higher priority.
                    \end{itemize}
              \item Schedule
                    \begin{itemize}
                        \item May 1st repositories and documentations walkthroughs.
                        \item May 2nd vulnerability analysis.
                        \item May 3rd - 8th operability testing.
                        \item May 9th performance testing.
                        \item May 10th – 11th Testing closure.
                    \end{itemize}
          \end{itemize}
\end{itemize}

\priloha{Test Plan No. 1}

\tab{Test plan No. 1}{tab:testplan1}{1}{|l|l|}{
    \hline
    Test plan ID & tp1 \\ \hline
    Revision & 1 \\ \hline
    Introducton &  Repositories walkthrough \\ \hline
    Test items & Operator's repositories \\ \hline
    Covered criteria  & C2, C4 \\ \hline
    Test type &	Static \\ \hline
    Test approach &	Repositories walkthrough \\ \hline
    Exit criteria & All metrics gathered  \\ \hline
    Delivarables & Sum of commits, sum of stars, sum of issues, sum of fixed issues.  \\ \hline
    Duration & 2 h for each Operator \\ \hline
    Reviewer & Miroslav Šiřina \\ \hline
    Start & May 1st \\ \hline
    Schedule & May 1st repositories walkthrough and test report. \\ \hline
}

\textbf{Test completition report}

Testing performed: Repositories walkthrough, repositories cloning, Percona Jira walkthrough

Deviations from planed testing: Percona is using Jira for tracking issues. To get issues Jira walkthrough was necessary. To count number of commits and get the date of first commit the repository cloning was necessary.

Test completion evaluation: The testing process was successful in gathering key data about the system despite deviations from the initial plan. The flexibility in testing procedures resulted in a more comprehensive evaluation and provided valuable insights into the system.

Factors that blocked progress: reposistory clonning, Jira walkthrough

Test Result Analysis: The tests provided valuable data about the state and history of repositories, as well as key insights into issue tracking.

Lessons Learned: The necessity to deviate from the initial test plan underlines the importance of flexibility in testing procedures. An adaptive approach can lead to a more thorough evaluation and better data collection.


\priloha{Test Plan No. 2}

\tab{Test plan No. 2}{tab:testplan2}{1}{|l|l|}{
    \hline
    Test plan ID & tp2 \\ \hline
    Revision & 1 \\ \hline
    Introducton & Checklist-based documentations walkthrough \\ \hline
    Test items & Operator's documentations \\ \hline
    Covered criteria  & C3 \\ \hline
    Test type &	Static \\ \hline
    Test approach &	Checklist-based Testing \\ \hline
    Exit criteria & All checklists completed  \\ \hline
    Delivarables & List of examples and checklist  \\ \hline
    Duration & 2 h for each Operator \\ \hline
    Reviewer & Miroslav Šiřina \\ \hline
    Start & May 1st \\ \hline
    Schedule & May 1st documentations walkthrough and test report. \\ \hline
}

\textbf{Checklist}
\begin{itemize}
    \item Instalation
    \item Minor upgrade to new version
    \item Major upgrade to new version
    \item Backup
    \item Restore
    \item Monitoring
    \item Vertical scaling
    \item Horizontal scaling
    \item Configuration Update
    \item Uninstall
    \item Training needed
\end{itemize}
\textbf{Test results}

PGO
Documentation
\sloppy
\url{https://access.crunchydata.com/documentation/postgres-operator/v5/}

Cluster creation

\url{https://access.crunchydata.com/documentation/postgres-operator/v5/tutorial/create-cluster/}

Minor upgrade to new version

\url{https://access.crunchydata.com/documentation/postgres-operator/5.3.1/tutorial/update-cluster/}

Major upgrade to new version

\url{https://access.crunchydata.com/documentation/postgres-operator/5.3.1/guides/major-postgres-version-upgrade/}

Backup

\url{https://access.crunchydata.com/documentation/postgres-operator/v5/tutorial/backup-management/}

Restore

\url{https://access.crunchydata.com/documentation/postgres-operator/5.3.1/tutorial/disaster-recovery/}

Monitoring

\url{https://access.crunchydata.com/documentation/postgres-operator/5.3.1/tutorial/monitoring/}

Vertical scaling

\url{https://access.crunchydata.com/documentation/postgres-operator/5.3.1/tutorial/resize-cluster/}

Horizontal scaling

\url{https://access.crunchydata.com/documentation/postgres-operator/5.3.1/tutorial/resize-cluster/}

Configuration Update

\url{https://access.crunchydata.com/documentation/postgres-operator/5.3.1/tutorial/customize-cluster/}

Uninstall

\url{https://access.crunchydata.com/documentation/postgres-operator/5.3.1/tutorial/delete-cluster/}

Notes: PGO use kustomize for customization of yaml manifests.

CNPGO

Documentation

\url{https://cloudnative-pg.io/documentation/1.19/}

Cluster creation

\url{https://cloudnative-pg.io/documentation/1.19/quickstart/#part-3-deploy-a-postgresql-cluster}

Minor upgrade to new version

\url{https://cloudnative-pg.io/documentation/1.19/rolling_update/}

Major upgrade to new version
Not found

Backup

\url{https://cloudnative-pg.io/documentation/1.19/backup_recovery/#scheduled-backups}

Restore

\url{https://cloudnative-pg.io/documentation/1.19/backup_recovery/#scheduled-backups}

Monitoring

\url{https://cloudnative-pg.io/documentation/1.19/monitoring/}

Vertical scaling

\url{https://cloudnative-pg.io/documentation/1.19/resource_management/#resource-management}

Horizontal scaling

\url{https://cloudnative-pg.io/documentation/1.19/resource_management/#resource-management}

Configuration Update

\url{https://cloudnative-pg.io/documentation/1.19/postgresql_conf/#postgresql-configuration}

Uninstall

\url{https://cloudnative-pg.io/documentation/1.19/cnpg-plugin/#destroy}

Notes: Uninstall example use cnpg plugin

SPGO
Documentation

\url{https://stackgres.io/doc/1.4/}

Cluster creation

\url{https://stackgres.io/doc/1.4/demo/quickstart/}

Minor upgrade to new version – It is mentioned in documentation but without example

\url{https://stackgres.io/doc/1.4/reference/crd/sgdbops/#major-version-upgrade}

Major upgrade to new version - It is mentioned in documentation but without example

\url{https://stackgres.io/doc/1.4/reference/crd/sgdbops/#minor-version-upgrade}

Backup

\url{https://stackgres.io/doc/1.4/tutorial/complete-cluster/backup-configuration/}

Restore

\url{https://stackgres.io/doc/1.4/runbooks/restore-backup/}

Monitoring

\url{https://stackgres.io/doc/1.4/install/prerequisites/monitoring/}

Vertical scaling

\url{https://stackgres.io/doc/1.4/tutorial/complete-cluster/instance-profile/}

Horizontal scaling

\url{https://stackgres.io/doc/1.4/tutorial/complete-cluster/create-cluster/}

Configuration Update

\url{https://stackgres.io/doc/1.4/tutorial/complete-cluster/postgres-config/}

Uninstall

\url{https://stackgres.io/doc/1.4/administration/uninstall/}

Notes: helm is needed to install monitoring

PPO
Documentation

\url{https://docs.percona.com/percona-operator-for-postgresql/index.html}

Cluster creation

\url{https://docs.percona.com/percona-operator-for-postgresql/gke.html#installing-the-operator}

Minor upgrade to new version

\url{https://docs.percona.com/percona-operator-for-postgresql/update.html?h=postgres+update#semi-automatic-upgrade}

Major upgrade to new version

\url{https://docs.percona.com/percona-operator-for-postgresql/update.html?h=postgres+update#semi-automatic-upgrade}

Backup

\url{https://docs.percona.com/percona-operator-for-postgresql/backups.html?h=backup#use-google-cloud-storage-for-backups}

Restore

\url{https://docs.percona.com/percona-operator-for-postgresql/backups.html?h=backup#use-google-cloud-storage-for-backups}

Monitoring

\url{https://docs.percona.com/percona-operator-for-postgresql/monitoring.html?h=version#installing-the-pmm-client}

Vertical scaling
Not found

Horizontal scaling

\url{https://docs.percona.com/percona-operator-for-postgresql/scaling.html?h=scale}

Configuration Update

\url{https://docs.percona.com/percona-operator-for-postgresql/options.html#creating-a-cluster-with-custom-options}

Uninstall
Not found

\textbf{Test completition report}

Testing performed: Checklist-based Testing

Deviations from planed testing: None

Test completion evaluation: The testing process was successful.

Factors that blocked progress: None

Test Result Analysis: The tests provided valuable data about the state of documentations, as well as key insights into Operators operation.

Lessons Learned: Future projects should be prepared for a high level of diversity in documentations. This might involve allocating more time for research or including personnel with a broader range of expertise.


\priloha{Test Plan No. 3}

\tab{Test plan No. 3}{tab:testplan3}{1}{|l|l|}{
    \hline
    Test plan ID & tp3 \\ \hline
    Revision & 1 \\ \hline
    Introducton & Vulnerability analysis of operators \\ \hline
    Test items & Operator deployed in the cluster \\ \hline
    Covered criteria  & C5 \\ \hline
    Test type &	Static \\ \hline
    Test approach &	Vulnerability analysis \\ \hline
    Exit criteria & Completed analysis  \\ \hline
    Tools & Trivy security scanner \\ \hline
    Delivarables & Vulnerability reports  \\ \hline
    Duration & 4 h for each Operator \\ \hline
    Reviewer & Miroslav Šiřina \\ \hline
    Start & May 2nd \\ \hline
    Schedule & May 2nd analysis and test report. \\ \hline
}

\textbf{Test completition report}

Testing performed: Vulnerability analysis

Deviations from planed testing: None

Test completion evaluation: The testing process was successful.

Factors that blocked progress: None

Test Result Analysis: The tests provided valuable data about vulnerabilities in Operators.

\priloha{Test Plan No. 4}

\tab{Test plan No. 4}{tab:testplan4}{1}{|l|l|}{
    \hline
    Test plan ID & tp4 \\ \hline
    Revision & 1 \\ \hline
    Introducton & Vulnerability analysis of operators \\ \hline
    Test items & Operator deployed in the cluster \\ \hline
    Covered criteria  & C5 \\ \hline
    Test type &	Static \\ \hline
    Test approach &	Vulnerability analysis \\ \hline
    Exit criteria & Completed analysis  \\ \hline
    Tools & Trivy security scanner \\ \hline
    Delivarables & Vulnerability reports  \\ \hline
    Duration & 4 h for each Operator \\ \hline
    Reviewer & Miroslav Šiřina \\ \hline
    Start & May 2nd \\ \hline
    Schedule & May 2nd analysis and test report. \\ \hline
}

\priloha{Test Plan No. 5}

\tab{Test plan No. 5}{tab:testplan5}{1}{|l|l|}{
    \hline
    Test plan ID & tp5 \\ \hline
    Revision & 1 \\ \hline
    Introducton & Performance analysis \\ \hline
    Test items & Operator deployed GKE in the cluster \\ \hline
    Covered criteria  & C1 \\ \hline
    Test type &	Dynamic \\ \hline
    Test approach &	Performance analysis \\ \hline
    Exit criteria & Completed analysis  \\ \hline
    Tools & PgBench Postgres benchmark tool  \\ \hline
    Delivarables & Performance reports  \\ \hline
    Duration & 4 h for each Operator \\ \hline
    Reviewer & Miroslav Šiřina \\ \hline
    Start & May 9th \\ \hline
    Schedule & May 9th analysis and test report. \\ \hline
}

\textbf{Test completition report}

Testing performed: Performance analysis

Deviations from planed testing: SPGO analysis too

Test completion evaluation: None

Factors that blocked progress: Deploying SPGO proved challenging due to the fact that SPGO requires a cluster profile to deploy the cluster, which will specify the allocated processor and memory. With the correct settings of these values, the cluster was unable to find suitable resources for SPGO. By gradually reducing these values, available resources were eventually found.

Test Result Analysis: The tests provided valuable data about performanc of Postgres deployed by Operators.

\textbf{Test results}

PGO

pgbench (15.2)

starting vacuum...end.

transaction type: <builtin: TPC-B (sort of)>

scaling factor: 1

query mode: simple

number of clients: 25

number of threads: 10

maximum number of tries: 1

number of transactions per client: 10000

number of transactions actually processed: 250000/250000

number of failed transactions: 0 (0.000\%)

latency average = 45.879 ms

initial connection time = 745.406 ms

tps = 544.913924 (without initial connection time)

pgbench (15.2)

starting vacuum...end.

transaction type: <builtin: TPC-B (sort of)>

scaling factor: 1

query mode: simple

number of clients: 25

number of threads: 10

maximum number of tries: 1

number of transactions per client: 10000

number of transactions actually processed: 250000/250000

number of failed transactions: 0 (0.000\%)

latency average = 46.016 ms

initial connection time = 688.436 ms

tps = 543.289895 (without initial connection time)

pgbench (15.2)

starting vacuum...end.

transaction type: <builtin: TPC-B (sort of)>

scaling factor: 1

query mode: simple

number of clients: 25

number of threads: 10

maximum number of tries: 1

number of transactions per client: 10000

number of transactions actually processed: 250000/250000

number of failed transactions: 0 (0.000\%)

latency average = 46.424 ms

initial connection time = 750.453 ms

tps = 538.511794 (without initial connection time)

CNPG

pgbench (15.2)

starting vacuum...end.

pgbench: error: client 6 script 0 aborted in command 4 query 0: FATAL:  query wait timeout

SSL connection has been closed unexpectedly

transaction type: <builtin: TPC-B (sort of)>

scaling factor: 1

query mode: simple

number of clients: 25

number of threads: 10

maximum number of tries: 1

number of transactions per client: 10000

number of transactions actually processed: 240000/250000

number of failed transactions: 0 (0.000\%)

latency average = 61.927 ms

initial connection time = 729.483 ms

tps = 403.698126 (without initial connection time)

pgbench: error: Run was aborted; the above results are incomplete.

command terminated with exit code 2

pgbench (15.2)

starting vacuum...end.

pgbench: error: client 21 script 0 aborted in command 4 query 0: FATAL:  query wait timeout

SSL connection has been closed unexpectedly

transaction type: <builtin: TPC-B (sort of)>

scaling factor: 1

query mode: simple

number of clients: 25

number of threads: 10

maximum number of tries: 1

number of transactions per client: 10000

number of transactions actually processed: 240000/250000

number of failed transactions: 0 (0.000\%)

latency average = 62.105 ms

initial connection time = 710.013 ms

tps = 402.544415 (without initial connection time)

pgbench: error: Run was aborted; the above results are incomplete.

command terminated with exit code 2

pgbench (15.2)

starting vacuum...end.

pgbench: error: client 1 script 0 aborted in command 4 query 0: FATAL:  query wait timeout

SSL connection has been closed unexpectedly

transaction type: <builtin: TPC-B (sort of)>

scaling factor: 1

query mode: simple

number of clients: 25

number of threads: 10

maximum number of tries: 1

number of transactions per client: 10000

number of transactions actually processed: 240000/250000

number of failed transactions: 0 (0.000\%)

latency average = 63.673 ms

initial connection time = 706.203 ms

tps = 392.629101 (without initial connection time)

pgbench: error: Run was aborted; the above results are incomplete.

command terminated with exit code 2

PPO

pgbench (15.2, server 14.7 - Percona Distribution)

starting vacuum...end.

transaction type: <builtin: TPC-B (sort of)>

scaling factor: 1

query mode: simple

number of clients: 25

number of threads: 10

maximum number of tries: 1

number of transactions per client: 10000

number of transactions actually processed: 250000/250000

number of failed transactions: 0 (0.000\%)

latency average = 62.272 ms

initial connection time = 707.856 ms

tps = 401.464378 (without initial connection time)

pgbench (15.2, server 14.7 - Percona Distribution)

starting vacuum...end.

transaction type: <builtin: TPC-B (sort of)>

scaling factor: 1

query mode: simple

number of clients: 25

number of threads: 10

maximum number of tries: 1

number of transactions per client: 10000

number of transactions actually processed: 250000/250000

number of failed transactions: 0 (0.000\%)

latency average = 63.769 ms

initial connection time = 690.586 ms

tps = 392.039997 (without initial connection time)

pgbench (15.2, server 14.7 - Percona Distribution)

starting vacuum...end.

transaction type: <builtin: TPC-B (sort of)>

scaling factor: 1

query mode: simple

number of clients: 25

number of threads: 10

maximum number of tries: 1

number of transactions per client: 10000

number of transactions actually processed: 250000/250000

number of failed transactions: 0 (0.000\%)

latency average = 64.467 ms

initial connection time = 574.644 ms

tps = 387.793393 (without initial connection time)

SPGO

pgbench (15.2, server 15.1 (OnGres 15.1-build-6.18))

starting vacuum...end.

transaction type: <builtin: TPC-B (sort of)>

scaling factor: 1

query mode: simple

number of clients: 25

number of threads: 10

maximum number of tries: 1

number of transactions per client: 10000

number of transactions actually processed: 250000/250000

number of failed transactions: 0 (0.000\%)

latency average = 87.906 ms

initial connection time = 60.527 ms

tps = 284.394442 (without initial connection time)

pgbench (15.2, server 15.1 (OnGres 15.1-build-6.18))

starting vacuum...end.

transaction type: <builtin: TPC-B (sort of)>

scaling factor: 1

query mode: simple

number of clients: 25

number of threads: 10

maximum number of tries: 1

number of transactions per client: 10000

number of transactions actually processed: 250000/250000

number of failed transactions: 0 (0.000\%)

latency average = 89.321 ms

initial connection time = 67.533 ms

tps = 279.890815 (without initial connection time)


pgbench (15.2, server 15.1 (OnGres 15.1-build-6.18))

starting vacuum...end.

transaction type: <builtin: TPC-B (sort of)>

scaling factor: 1

query mode: simple

number of clients: 25

number of threads: 10

maximum number of tries: 1

number of transactions per client: 10000

number of transactions actually processed: 250000/250000

number of failed transactions: 0 (0.000\%)

latency average = 80.863 ms

initial connection time = 74.577 ms

tps = 309.164610 (without initial connection time)

pgbench (15.2, server 15.1 (OnGres 15.1-build-6.18))

% ============================================================================ %
