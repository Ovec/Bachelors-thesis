% ============================================================================ %
% Encoding: UTF-8 (žluťoučký kůň úpěl ďábelšké ódy)
% ============================================================================ %

% ============================================================================ % 
\nn{Introduction}
It is essential for the database server to be as close as possible to the applications that are using it. This reduces the number of men in the middle between the database and the application, which reduces database access latency and thus reduces overall application latency and increases security. The mass migration of applications to Kubernetes clusters implies a necessary shift of Postgres to Kubernetes. This thesis defines Postgres, Kubernetes, and their Operators. It then further describes the lifecycle of a Postgres cluster, and searches for Operators capable of managing this lifecycle. It establishes metrics by which it tests and evaluates these Operators.  The result of this thesis is the recommendation of a suitable Operator based on the defined metrics.

TBD - remove

The cloud has made our work easier. We no longer have to physically connect new machines to the network, configure network connections, add disks, or even plug in virtual ones. Kubernetes, together with the cloud, can automatically allocate new resources for our applications and, thanks to operators, can even create entire database clusters with high availability. Thanks to operators, it can also automatically set up scaling or backups. It can even restore an entire database system from a backup. This paper is focused on Kubernetes operators for the popular Postgres database management system. Its goal is to find Operators for Postgres. To evaluate their pros and cons. To test them and recommend the best one.

TBD - remove end

TBD - switch to Czech layout

TBD - Motto

TBD - Czech abstract, English abstract

TBD - Poděkování, motto

TBD - adjust the test theory to test process

TBD - navrhout doporučení, rozhodovací strom

% ============================================================================ %
\cast{Theory}
\sloppy
\n{1}{Background}
This chapter introduces the key technologies used in this thesis including Postgres, Kubernetes, and Kubernetes Operators.
\n{2}{Postgres}
TBD: Connect image with text

PostgreSQL is a powerful object-relational database management system (ORDBMS) derived from the POSTGRES package written at the University of California at Berkeley. \cite{docuPgwhatIsPg} \cite{pg14introduction} The first version of POSTGRES was released in June 1989. POSTGRES has been used in many applications, including financial data analysis systems, asteroid tracking databases, medical information database, and several geographic information systems. The size of external community users has nearly doubled by 1993. \cite{docuPgBriefHistory}

POSTGRES was using its POSTQUEL query language from version, until Andrew Yu and Jolly Chen introduced SQL to POSTGRES in 1995. The name has changed to Postgres95. Postgres95 was completely ANSI C code reduced by 25 \% and was 30 – 50 \% faster than Postgres 4.2.  \cite{docuPgBriefHistory}

It was clear by 1996 that the name would not stand the test of time therefore it has been renamed to PostgreSQL. As stated by PostgreSQL documentation \cite{docuPgBriefHistory}: “Many people continue to refer to PostgreSQL as “Postgres” (now rarely in all capital letters) because of tradition or because it is easier to pronounce. This usage is widely accepted as a nickname or alias.“ This thesis will use Postgres as an alias for PostgreSQL as well.

More than 30 years after the first version Postgres has been considered the most used ORDBMS for professional developers by Stack Overflow survey \cite{so2022survey}. According to Riggs and Ciolli \cite{pg14introduction}: “The PostgreSQL feature set attracts serious users who have serious applications. Financial services companies may be PostgreSQL's largest user group, although governments, telecommunication companies, and many other segments are strong users as well.“ It is fully ACID compliant \cite{juba2015learningTransactionIsolation} and supports many kinds of data models such as relational, document, and key/value. \cite{pg14introduction}

\obr{Postgres Architecture \cite{HusseinMediumPostgres} }{fig:Postgres_architecture}{1}{graphics/Postgres_architecture.png}


\n{3}{Write Ahead Log}
Write-ahead Logging (WAL) used by Postgres is a standard technique to ensure data integrity. Its main concept is that changes in data files (where tables and indexes are stored) must only be written after they are logged (saved to a log file). That means the database is updated after the changes are written to disk. In the event of a system crash, all transactions will be recovered from the disk. \cite{docuPgWal}

Although WAL is primarily designed for recovery after a database server crash, its design also allows any changes to the database server state to be replayed backward. A copy of the log is also a form of backup. Thus, for recovery to a point in time, only logs that have been saved to that point in time can be restored. This technique is called Point-In-Time Recovery (PITR). \cite{DocuPgPITR} These log files can also be streamed to other nodes to serve as a replica or remote backup. \cite{pg14replication}

TBD: describe synchronous and asynchronus replication

\n{3}{Backup and restore}
A full set of backup commands is included in Postgres. Among the simple backup commands are pg\_dump and pg\_dumpall, which enable one or more databases to be saved in SQL format. A wide range of configuration options are available for these commands, including compression for large databases or exporting only the database schema. To restore a database from a file at a later time, the psql command can be used, which is capable of restoring a database from its dump. \cite{DocuPgDump} These commands are also helpful with migration from one major Postges version to another because the dumped files are plain SQL commands.

The backup options in Postgres are quite limited. Postgres allows to set up of a backup command that runs after the next log file is created, database dumps, and log streaming. For more advanced backup techniques, additional software such as PgBackRest must be used. \cite{DocuPgPITR}

\n{4}{PgBackRest}
PgBackRest is a reliable and simple backup and restore solution that provides many features on top of classic Postgres backup and restore tools like parallel backup options with compression, local or remote backups, cloud backup (S3. Azure and Google Cloud), or backup encryption. Full, incremental, or differential backup is also supported. \cite{PGbackRest}

TBD: why is it here? Conect to crunhy and operatores


\n{3}{High Availability}
The basic structure of a database cluster consists of one or more database servers, which can be called nodes. In Postgres there are two types of nodes, Primary node and Standby node.  A Primary node is such a node that allows reading and writing information. The newly written information is then streamed to the Standby nodes. Standby nodes are read-only, they do not allow writing. \cite{pg14replication}

Achieving high availability with Postgres is possible by using more than one node in the cluster. Two options are possible here. A single Primary node option, where the Primary node is read and write enabled, and the other nodes are Standby nodes. If the Primary node is unavailable, then the Standby node is promoted to the Primary node. This event is called failover. In this variant, the Primary node streams the logs to the Standby nodes. The second option is to use multiple Primary nodes. However, conflicts can occur because all Primary nodes allow concurrent writes. \cite{docuPgHA}
\n{4}{Patroni}
Since Postgres does not provide any software that can detect that a node is unavailable, it is necessary to use software outside of Postgres \cite{docuPgFailover}, such as Patroni.
Patroni is a popular open-source tool created by Zalando to achieve high availability of Postgres clusters. Patroni uses a distributed configuration source such as ZooKeeper, Etcd, Consul, or Kubernetes for its operation. Patroni can automatically adjust the settings of all managed nodes, therefore it can automate failover and make it seamless. \cite{PalarkMigratingPg} \cite{PatroniDocu}


\n{3}{Load Balancing and Connection Pooling}
Using more than one node allows to direct traffic to a node that is less busy and thus achieve load balancing. Postgres doesn't come with any software that allows splitting the load on different nodes, so it is necessary to use an external load balancer such as HA Proxy or pgBouncer. The load balancer then acts as an intermediary between the database and the client and directs the traffic to the available nodes according to the set rules. These load balancers also enable connection pooling which is a technique for managing and reusing database connections to increase performance and reduce overhead. Connection pooling involves creating a pool of pre-created connections that can be shared and reused by multiple client requests, instead of creating a new connection for each request. This removes the overhead of creating a new process each time a client connects to Postgres and allows the client to use resources that would otherwise be used to service multiple requests (or complete them faster). \cite{PerconaBlogConnectionPooling}

\n{2}{Kubernetes}
Kubernetes, also known as K8s, is an open-source platform for automating deployment, scaling, and management of containerized applications. It provides a way to manage and orchestrate containers, which are units of software that package up an application and its dependencies into a single, isolated package that can run consistently on any infrastructure. \cite{vayghan2019Kubernetes}

As described by Kubernetes Documentation \cite{docuKubeComponents} Kubernetes provides several key features, including:
\begin{itemize}
    \item \textbf{Service discovery:} A container can be exposed by Kubernetes either through its DNS name or its own IP address.
    \item \textbf{Load balancing:} In the case of high traffic to a container, stability of the deployment can be ensured by Kubernetes load balancing and distributing the network traffic.
    \item \textbf{Storage Orchestration:} Storage orchestration in Kubernetes allows for the automatic mounting of a storage system of choice, including local storage, public cloud providers, and others.
    \item \textbf{Automated rollouts and rollbacks:} The desired state of deployed containers can be described using Kubernetes, and the actual state can be changed to the desired state at a controlled rate. For instance, the automation of Kubernetes can be utilized to create new containers for the deployment, remove existing containers, and transfer all their resources to the newly created container.
    \item \textbf{Automatic bin packing:} A cluster of nodes for running containerized tasks is provided to Kubernetes. The amount of CPU and memory required by each container is specified to Kubernetes. The optimal utilization of resources can be achieved by Kubernetes fitting the containers onto the nodes.
    \item \textbf{Self healing:} Containers that fail are restarted by Kubernetes, those that do not respond to the user-defined health check are replaced or killed, and they are not advertised to clients until they are deemed ready to serve.
    \item \textbf{Secret and configuration management:} Sensitive information, such as passwords, OAuth tokens, and SSH keys, can be stored and managed by Kubernetes. The deployment and updating of secrets and application configuration can be done without the need to rebuild container images and without the exposure of secrets in the stack configuration.
\end{itemize}
\n{3}{Kubernetes Components}
\label{chap:kubeComponents}
Kubernetes cluster is composed of a set of worker machines that run containerized applications called nodes. Each cluster must have at least one node. \cite{docuKubeComponents}
\obr{The components of a Kubernetes cluster \cite{docuKubeComponents}}{}{1}{graphics/Kubernetes_cluster_components.png}

The Kubernetes control plane is the management system of a Kubernetes cluster, responsible for maintaining the desired state of the cluster. It consists of multiple components that work together to manage the cluster and its resources, including pods, services, and volumes. The key components of control plane are \cite{masteringKubernetesConcepts}:
\begin{itemize}
    \item \textbf{kube-APIserver:} Acts as the front-end for the Kubernetes API and exposes the API to other components. \cite{docuKubeComponents}
    \item \textbf{Etcd:} Highly available distributed key-value store that serves as the backing store for the cluster's configuration data. \cite{Dobies2020}
    \item \textbf{kube-scheduler:} Assigns work to nodes in the cluster, such as scheduling pods to run on nodes. \cite{kubeUpAndRunningPods}
    \item \textbf{kube-controller-manager:} Monitors the cluster's state and makes adjustments as necessary to maintain the desired state. \cite{masteringKubernetesConcepts}
    \item \textbf{cloud-controller-manager:} Manages cloud-related tasks such as node creation and management, volume management, and load balancing, allowing the other components of the control plane to focus on their specific responsibilities. Cloud manager is optional. Can be avoided when Kubernetes not used in cloud. \cite{docuKubeComponents}
\end{itemize}
\textbf{Node components:}
Node components in a Kubernetes cluster run on each node and provide crucial functionality for the operation of containers on that node. \cite{docuKubeComponents}
\begin{itemize}
    \item \textbf{kubelet:} Is responsible for communicating with the control plane and ensuring that containers are running and healthy. \cite{kubeUpAndRunning}
    \item \textbf{kube-proxy:} Is responsible for maintaining network rules on the nodes, allowing network communication to the containers. It enables the containers in a pod to communicate with other containers and the outside world, and performs tasks such as load balancing and traffic routing. \cite{kubeUpAndRunning}
    \item \textbf{container runtime:} Is responsible for running containers. \cite{docuKubeComponents}
\end{itemize}

\n{3}{Kubernetes Concepts}
\label{chap:kubeConcepts}
\textcolor{cyan}{RepliceSet extension - Operators p. 28 (Replica is general and appllication agnostic) }

Pod is the smallest deployable unit that can be created in Kubernetes. \cite{docuKubePods} A Pod in Kubernetes is comprised of multiple containers and storage volumes that are run together within the same execution environment. As a result, all containers included in a single Pod will always run on the same machine. \cite{kubeUpAndRunningPods}
A Pod's specifications are outlined in a Pod manifest, which is simply a JSON or YAML text file that represents the Kubernetes API object. Kubernetes follows a declarative configuration approach, where the system's desired state is defined in a configuration file, and the service then implements the necessary changes to make the desired state a reality. \cite{docuKubeStaticPod}

ReplicaSet’s purpose is to ensure a consistent number of replica Pods are running at all times. It is commonly used to guarantee a specified number of identical Pods are available. However, a Deployment is a more advanced concept that oversees ReplicaSets and provides a more streamlined way to make updates to Pods. It also offers additional features. As a result, it's advisable to use Deployments instead of directly utilizing ReplicaSets, unless you have specific update requirements or don't need to make updates at all. \cite{docuKubeReplicaset}

Service is an abstraction layer and defines a group of Pods and the method to access them (often referred to as a micro-service). The group of Pods targeted by a Service is usually specified through a selector. The Service abstraction makes this possible by enabling the decoupling of components. \cite{docuKubeSevice} Kubernetes includes built-in service discovery mechanisms. When a service is created in Kubernetes, it is automatically assigned an IP address and DNS name. Clients and other services can use this name or address to access the service within the Kubernetes cluster. \cite{docuKubeSevice}

Containers and pods in Kubernetes are ephemeral. When a container is terminated, any data it has written to its own filesystem is lost. In Kubernetes, storage is represented by a basic abstraction called "volumes". Containers use these volumes by binding them to their respective pods, and can then access the storage regardless of its physical location as if it were a part of their local filesystem. \cite{masteringKubernetesStorage}

Kubernetes version 1.5 came with a new object called StatefulSet that allows a set of stateful pods to be deployed and managed. Each pod has a unique, stable network identity and a persistent storage volume. This enables stateful applications like databases to be run on Kubernetes. Advantages of using StatefulSets include predictable naming schemes, ordered pod creation and deletion, and unique persistent storage. \cite{docuKubeStatefulSet} \cite{githubKube15}

In version 1.7, Kubernetes introduced the Custom Resources extension to its API. \cite{githubIBMCr} This extension allows Kubernetes to use user-defined resources that are not native to Kubernetes as if they were native. \cite{NewStackCRDs} Custom resources (CR) is an extension to the Kubernetes API that extends the deployment with additional parameters that are not part of it. CR stores these parameters and allows the API server to access them just like the native Kubernetes parts. CR is created in the Kubernetes cluster using a definition called Custom Resource Definition (CRD). \cite{OperatorsAtK8sIface}

Kuberentes controllers are control loops\footnote[1]{A control loop is a process that continuously monitors the state of a system, compares it to a desired state, and makes adjustments to bring the system closer to the desired state.} that constantly check the state of their controlled objects. If the controlled objects are not in the desired state, the controller performs actions to get the controlled objects into that state. For example, restart a crashed node, add a new replica, modify settings, etc. \cite{docuKubeControllers}
However, to work with CR, custom controllers that can work with these resources must be created, these controllers are called Custom Controllers. \cite{docuKubeCR}


\obr{Kubernetes controller \cite{OperatorWhitepaper} }{}{1}{graphics/Kubernetes_controller.png}

\textcolor{red}{TBD - show that Kubernetes can run stateless very well, maybe from Operator book}


\textcolor{cyan}{TBD - Read https://containerjournal.com/kubecon-cnc-eu-2022/why-run-postgres-in-Kubernetes/}

\textcolor{cyan}{TBD - Read data on Kubernetes https://dok.community/dokc-2021-report/}


\n{2}{Running Postgres in Kubernetes}
\label{chap:postgresInKube}
Kubernetes cannot know all complex stateful applications, which can contain a large number of nodes and have a wide range of uses while remaining general-purpose. The goal of Kubernetes is to provide an abstraction covering basic application concepts and providing options for extensions for more complex applications and their specific operations. Kubernetes cannot and should not know all the possible settings and operations that, for example, a Postgres cluster needs to run. \cite{OperatorsTeaches}

The easiest way to run Postgres in Kubernetes is through the StatefulSet just mentioned. This StatefulSet can start a Postgres pod, create a persistent volume, and connect this volume to the pod. A stateful set can do this for all replicas set in its configuration. It can also scale up or down. Unfortunately, however, all independent Postgres instances created by StatefulSet controller are not synchronized in any manner.

This basic setup may be sufficient for running a single node, but it is no longer sufficient fro managing the whole Postgres lifecycle. For managing whole Postgres lifecycle it is necessary to install other applications in the Kubernetes cluster and then configure the entire Postgres cluster to work with them. This represents a large amount of work and subsequent maintenance that Kubernetes Operators can facilitate.

\n{2}{Database System Lifecycle}
\label{chap:lifecycle}
The database system itself is a software like any other. It is therefore also subject to the same life cycle as software.
As depicted in figure \ref{fig:applifecycle} application lifecycle consists of three main parts. It is the governance part, development, and operations. For this thesis, only the operations part is relevant because it is the only part we are able to control.
\obr{Application Life Cycle \cite{ALM}}{fig:applifecycle}{1}{graphics/aplication-lifecycle.png}

Operation is the process of running and managing the application, which starts with deployment and continues until the application is taken out of service. This aspect of the application lifecycle management covers the release of the application into production, ongoing monitoring, and other related tasks. \cite{ALM}

Therefore the complete database life cycle can be outlined by following events:
\begin{itemize}
    \item System installation
    \item System upgrade to a newer version (major and minor)
    \item System backup
    \item System restore
    \item System monitoring
    \item System scaling (vertical and horizontal)
    \item System configuration update
    \item System uninstall
\end{itemize}

\pagebreak
\n{2}{Operators}
\label{chap:operators}
As described in chapter~\ref{chap:postgresInKube}, Kubernetes can run stateless applications very well. But its general purpose makes running complex stateful applications on top of it quite challenging.


This has changed in 2016 when CoreOS came up with Operators as a way to deploy complex applications with state such as databases, caches, or monitoring systems. \cite{IArchiveCOSOperators}

An operator is a special kind of software that extends the Kubernetes API and has a particular knowledge of managed resource that Kubernetes does not have. The Operator also serves as a packaging mechanism for distributing applications including their dependencies in Kubernetes. The Operator can manage, restore, update or monitor the resource. It can also manage very complex applications. The Kubernetes Operator thus replaces the human operator after which it is named, who would otherwise take care of these tasks. \cite{OperatorsPreface} \cite{IArchiveCOSOperators}

\obr{Definition of Kubernetes Operator \cite{IArchiveCOSOperators}}{}{1}{graphics/coreos_operator.png}

CoreOS demonstrated the use of its Operator on Etcd (described in the Kubernetes Components chapter). When new Etcd nodes are created, it is necessary to give them a DNS names and use the Etcd cluster management tools to add the new nodes to an existing cluster. CoreOS has automated these tasks with the Etcd Operator so that all that is required is to increase the number of replicas in the Operator CRD and the Etcd Operator will perform these tasks instead of a human operator. \cite{IArchiveCOSOperators}
By embedding the human operator's operational knowledge into the code, this ensures that these tasks are repeatable, testable and upgradable. It also ensures that the necessary operations are always performed, executed in the order in which they are supposed to be performed, and none are skipped. This reduces the number of hours spent on dull but essential work such as backups. \cite{OperatorWhitepaper}

As described by Operator White Paper \cite{OperatorWhitepaper} and depicted in figure \ref{fig:operatorPatern}, Operator consists of the following parts
\begin{itemize}
    \item The managed application or infrastructure
    \item Software that has some specific knowledge of the managed application or infrastructure and allows the user to declaratively set the desired state
    \item Custom Controller, which is responsible for achieving the desired state
\end{itemize}

\obr{Operator pattern \cite{OperatorWhitepaper} }{fig:operatorPatern}{1}{graphics/operator_patern.png}

Like human operators, Kubernetes Operators can have a level of manual skill ranging from basic software installation and setup skills to a high level where they can scale software vertically or horizontally to automatically change the configuration or detect abnormalities. All Operator maturity levels are depicted in the figure \ref{fig:operatorCapabilities}. The highest level can only be reached by programming the Operator in the GO programming language or by using the Ansible automation tool. \cite{OperatorsOframework}

\obr{Operator maturity levels described by Operator Framework \cite{OFrameworkMaturity}}{fig:operatorCapabilities}{1}{graphics/operator_capabilities.png}

As stated in the Operator white paper, \cite{OperatorWhitepaper} the Operator should be able to cover the complete life cycle of the managed resource as defined in the previous chapter without the need for external installation or upgrade intervention. Specifically as follows:
\begin{itemize}
    \item Install or take ownership of the controlled application.
    \item Upgrade the managed application, including the monitoring of the upgrade process. It should also be able to roll back in case of failure. He should record the status of the upgrade.
    \item Back up the managed application and log when the application was last backed up and the status of that backup.
    \item Restore the application from the backup.
    \item Provide monitoring of the managed application.
    \item Scale the application.
    \item Automatically adapt the configuration of the application.
    \item Uninstall or disconnect from the application.
\end{itemize}

These are all capabilities that an Operator should have at the highest level No. 5 - Autopilot. For lifecycle management described in previous chapter, the minimum level of Operator capabilities must be at least level No. 4 - Deep Insights with an option to scale.

The Kubernetes cluster is divided into individual namespaces that separate the objects and names in the cluster and can have constraints applied to them. This partitioning makes it easier to share the cluster between users or entire teams. The object name must be unique within a namespace, but not between namespaces.  An operator usually operates in its own namespace so it has a Namespace Scope, but it can also operate in the whole cluster in which case it will be a Cluster Scope Operator. Namespace Scope Operators are more flexible and easier to upgrade due to their independence from the rest of the cluster. Operator rights are further restricted by the so-called Role-Based Acceess Control (RBAC), which grants the rights assigned to the Operator. \cite{ OperatorsAtK8sIface}

The following options are advised by the Operator white paper \cite{OperatorWhitepaper} in case the Operator is to be used for controlling the resource:
\begin{itemize}
    \item	Consultation with the creator of the resource to be controlled about the possibilities of using the Operator.
    \item	The search for public Operator registries that provide a platform for publishing Operators and the underlying documentation.
    \item	The creation of own Operator.
\end{itemize}

\n{1}{Thesis objective}
The objective of this thesis is to conduct an extensive evaluation of various Kubernetes Operators available for Postgres lifecycle management.

The ultimate aim is to provide clear and comprehensive recommendations on which Operators are best suited to meet different stakeholder needs and preferences. This thesis intends to serve as a valuable resource that can guide stakeholders in making informed decisions about choosing the right Operator for their specific operational context and requirements.

Furthermore, it aims to contribute to the broader knowledge base about Kubernetes Operators and their application in managing Postgres databases in a cloud-native environment.


\n{1}{Resource questions}
Having defined the thesis objective and established a thorough understanding of PostgreSQL, Kubernetes, the PostgreSQL lifecycle, and Operators in previous chapters, we are now in a position to formulate research questions.
These questions are designed to enable a deeper exploration into the complexities of managing Postgres in a Kubernetes environment via Operators.

\begin{enumerate}
    \item What Operators exist for lifecycle management of Postgres in Kubernetes?
    \item What metrics are suitable for comparing Opereators for lifecycle management in Kubernetes?
    \item What approach should be taken to determine the degree to which the metrics are met?
    \item How do the Operators perform when evaluated according to the chosen metrics?
\end{enumerate}

\n{1}{Operators for Lifecycle Management in Kubernetes}
\label{chap:searchForOperators}
This chapter aims to answer the first research question: 'What Operators exist for lifecycle management of Postgres in Kubernetes?'"
As recommended in chapter~\ref{chap:operators} the selection of the Operator should first be consulted with the manufacturer of the controlled source. Postgres offers the following Kubernetes Operators in its software catalog \cite{docuPgSwCatalogue}: CloudNativePG, EDB Postgres for Kubernetes a Kubegres.

The next step involved a search of the Operators' registers. In particular the Operator Hub. \cite{OperatorHubPGSearch} Operator Hub presents nine operators with varying levels of capabilities, including Crunchy Postgres for Kubernetes by Crunchy Data, EDB Postgres for Kubernetes by EnterpriseDB Corporation, Ext Postgres Operator by movetokube.com, Percona Operator for PostgreSQL by Percona, Postgres-Operator by Zalando SE, Postgresql Operator by Openlabs, PostgreSQL Operator by Dev4Ddevs.com and StackGres by OnGres.

A deeper internet search revealed Stolon Operator. \cite{PalarkComparingKubernetes}

Of the thirteen operators available, only five meet the minimum capability requirement of Deep Insight, namely: Crunchy Postgres for Kubernetes, EDB Postgres for Kubernetes, Percona Operator for PostgreSQL, CloudNativePG Operator, and StackGres Operator. As a result, only these five will be subjected to deeper research, testing, and evaluation.

\pagebreak
\n{2}{Crunchy Postgres for Kubernetes}
Crunchy Postgres for Kubernetes (PGO) is a Postgres Operator provided by Crunchy Data, which offers a declarative solution for the management of PostgreSQL clusters, with a focus on automation.
Crunchy Data is a company that specializes in providing open-source software solutions for Postgres. The company also provides a range of support, consulting, and training services to help organizations implement and optimize their Postgres deployment. \cite{Crunchy}

PGO’s capabilities are the following:
\begin{itemize}
    \item \textbf{Postgres Cluster Provisioning:} PGO is able to create \cite{CrunchyDocCreate}, update \cite{CrunchyDocUpdate} or delete Postgres cluster \cite{CrunchyDocDelete}
    \item \textbf{High Availability:} High availability is achieved by adding additional nodes. PGO uses a synchronous replication technique with Primary and Standby architecture. \cite{CrunchyDocHA}
    \item \textbf{Postgres updates:} PGO is able to apply minor patches \cite{CrunchyDocMinorUpdates}, and major upgrades since version 5.1. \cite{CrunchyBlogUpdates}
    \item \textbf{Backups:} PGO backup capabilities features: automatic backup schedules, backup to multiple locations, backup to cloud providers (AWS S3, Google Cloud Storage, Azure Blob), ad hoc backups, backup compression, and backup encryption. \cite{CrunchyDocBackups}
    \item \textbf{Disaster Recovery:} PGO is capable of Point In Time recovery, in place Point in Time Recovery, restore of an individual database. \cite{CrunchyDocDisasterRecovery}
    \item \textbf{Cloning:} PGO is able to clone cluster. \cite{CrunchyDocDisasterRecovery}
    \item \textbf{Monitoring:} Monitoring is provided by Prometheus, Grafana, and Alertmanager. \cite{CrunchyDocMonitoring}
    \item \textbf{Connection Pooling:} PgBouncer connection pooler from Postgres is part of PGO. \cite{CrunchyDocConnectionPooling}
    \item  \textbf{Customization:} PGO provides a wide area of Postgres customization. \cite{CrunchyDocCustomisation}
\end{itemize}


PGO consists of the following key components \cite{CrunchyPGOGit}:
\begin{itemize}
    \item High Availability: Patroni
    \item Backups: PgBackRest
    \item Connection Pooler: PgBouncer
    \item Monitoring: PgMonitor, Prometheus, Grafana, and Alertmanager
\end{itemize}
\obr{PGO’s architecture \cite{CrunchyDocArchitecture}}{}{1}{graphics/pgo_architecture.png}

The current stable version of PGO is 5.3.1 was released on 17th February 2023. \cite{CrunchyV531releaseNotes}

PGO is distributed under the Apache License 2.0, an open-source license that allows for both commercial and non-commercial use. With regards to capability, PGO is considered to have the highest capability level, labeled as Autopilot. \cite{OperatorHubCrunchy}

\n{2}{EDB Postgres for Kubernetes}
The EDB Postgres for Kubernetes (EDBO) is a fully supported operator that has been designed, developed, and maintained by EnterpriseDB Corporation. It provides comprehensive coverage of the entire lifecycle of highly available PostgreSQL database clusters with a Primary/Standby architecture, utilizing native streaming replication. The operator is based on the open-source CloudNativePG operator and offers additional benefits. \cite{OperatorHubEDB}

EDBO is distributed under the EDB Limited Usage License Agreement, a proprietary license that is specific to software provided by EnterpriseDB Corporation. A license key is always required for the operator to work longer than 30 days. \cite{EDBdocuLicence} Due to the restrictive nature of the license EDBO will no longer be subject to testing and evaluation but its key component CloudNativePG will.

\n{2}{CloudNativePG}
The CloudNativePG operator (CNPGO) is an operator that is available as an open-source solution and aims to manage PostgreSQL workloads across various Kubernetes clusters running in private, public, hybrid, or multi-cloud environments. The Operator aligns with DevOps principles and concepts like immutable infrastructure and declarative configuration. \cite{CNPGdocu}

Initially developed by EDB, CNPGO was later made available to the public as an open-source software under the Apache License 2.0. In April 2022, the project was submitted to CNCF Sandbox for further development and community engagement. \cite{CNPGdocu}

CNPGO’s capabilities are the following:
\begin{itemize}
    \item \textbf{Postgres Cluster Provisioning:} CNPGO is able to create, update or delete Postgres cluster. \cite{CNPGdocuCapabilityLevels}
    \item \textbf{High Availability:} High availability is achieved by adding additional nodes. PGO uses a synchronous replication technique with Primary and Standby architecture. \cite{CNPGdocuReplication}
    \item \textbf{Direct database imports:} CNPGO provides direct database import from remote Postgres server by using pg\_dump and pg\_restore even on different Postgres versions. \cite{CNPGdocuDatabaseImports}
    \item \textbf{Postgres updates:} CNPGO is able to apply minor patches. \cite{CNPGdocuUpdates} Major updates are possible by Direct database imports\footnote[2]{Due to its nature Direct database imports cannot be considered as major upgrade option.}.
    \item \textbf{Backups:} CNPGO backup capabilities features: automatic backup schedules, backup to multiple locations, backup to cloud providers (AWS S3, Google Cloud Storage, Azure Blob), on-demand backups, and backup encryption \cite{CNPGdocuBackup}\cite{CNPGdocuTDE}. Due to EDB’s backup software Barman backup compression is available also. \cite{CNPGdocuBackup}
    \item \textbf{Disaster Recovery:} CNPGO is capable of Point In Time recovery. \cite{CNPGdocuBackup}
    \item \textbf{Cloning:} CNPGO is able to create cluster replicas. \cite{CNPGdocuReplication}
    \item \textbf{Monitoring:} Monitoring can be provided by the additional installation of Prometheus, and Grafana, and Alertmanager. \cite{CNPGdocuQuickstart}
    \item \textbf{Connection Pooling:} Provided by native Postgres pooler PgBouncer. \cite{CNPGdocuConnectionPooling}
    \item \textbf{Customization:} CNPGO provides a wide area of Postgres customization such as max parallel workers tuning or WAL configuration \cite{CNPGdocuConfiguration}
\end{itemize}

CNPGO consists of the following key components \cite{PostgresOnKubernetes} \cite{CNPGdocuQuickstart}:
\begin{itemize}
    \item High Availability: Postgres instance manager
    \item Backups: Barman
    \item Connection Pooler: PgBouncer
    \item Monitoring: Prometheus, Grafana, and Alertmanager
\end{itemize}

\obr{CNPGO’s architecture \cite{CNPGdocuConnectionPooling}}{}{1}{graphics/CNPGO_architecture.png}

The current major stable version of CNPGO is 1.20.0 was released on 27th April 2023. \cite{CNPGreleases} CNPGO is distributed under the Apache License 2.0 open-source license. CNPGO is considered to have the highest capability level, labeled as Autopilot. \cite{CNPGdocu}

\n{2}{StackGres Operator}
StackGres (SPGO) is a comprehensive distribution of Postgres for Kubernetes, delivered in a user-friendly deployment package. The distribution includes a set of Postgres components that have been carefully selected and optimized to work seamlessly with each other. \cite{SPGOgitlab}

SPGO is developed by OnGres that was established as a result of years of experience in working with and creating products based on Postgres and supporting clients with their Postgres infrastructures. Postgres databases are at the heart of the company's business, as the name suggests. \cite{OnGres}

SPGO’s capabilities are the following \cite{OnGres}:
\begin{itemize}
    \item \textbf{Postgres Cluster Provisioning:} SPGO is able to create, update or delete Postgres cluster.
    \item \textbf{High Availability:} High availability is achieved by adding additional nodes with Primary and Standby architecture.
    \item \textbf{Postgres updates:} SPGO is able to apply minor patches. Major updates are possible by SGDbOps \cite{SPGODocuMajorUpdates}.
    \item \textbf{Backups:} SPGO backup capabilities features: automatic backup schedules, backup to multiple locations, backup to cloud providers (AWS S3, Google Cloud Storage, Azure Blob)
    \item \textbf{Disaster Recovery:} SPGO is capable of Point In Time recovery.
    \item \textbf{Cloning:} SPGO is able to create cluster replicas.
    \item \textbf{Monitoring:} Monitoring is provided by Prometheus, Grafana, and Alertmanager.
    \item \textbf{Connection Pooling:} Is provided by PgBouncer.
    \item \textbf{Customization:} SPGO provides a wide area of Postgres customization such as WAL configuration, archive mode, vacuum, etc. \cite{SPGODocuCustomization}
    \item \textbf{Mamagement Console:} SPGO provides a fully featured management web console.

\end{itemize}

SPGO consists of the following key components \cite{PostgresOnKubernetes}:
\begin{itemize}
    \item High Availability: Patroni
    \item Backups: WAL-G
    \item Connection Pooler: PgBouncer
    \item Monitoring: Prometheus, Grafana, and Alertmanager.
\end{itemize}

\obr{SPGO’s architecture \cite{SPGOdocuArchitecture}}{}{1}{graphics/SPGO_architecture.png}

The current stable version of SPGO is 1.4.3\footnote[3]{Version 1.5.0 has also been released, it is not yet production-ready yet. Therefore will not be tested or evaluated.} was released on 24th January 2022. \cite{SPGOgitlabChangelog} SPGO is distributed under the AGPL3 open-source license. \cite{SPGODocuLicence}
With regards to capability, SPGO is considered to have the second highest capability level, labeled as Deep Insights. \cite{OperatorHubStackgres}

\n{2}{Percona Operator for PostgreSQL}
\label{chap:ppo}

Percona is a company that provides services and solutions for open-source database technologies. It offers expertise, support, and software for MySQL, MongoDB, and PostgreSQL. The company's offerings help organizations manage their open-source databases and ensure optimal performance, security, and scalability. \cite{Percona}

Percona Operator for PostgreSQL (PPO) is based on Crunchy Postgres for Kubernetes. Percona forked PGO v 4.7 and has added enhancements for monitoring, upgradability, and flexibility. \cite{PerconaBlogProsAndCons}

Differences between PGO and PPO are the following:
\begin{itemize}
    \item \textbf{Postgres updates:} PPO provides automatic Postgres updates for minor and major versions of Postgres.  \cite{PerconaDocuUpdate}
    \item \textbf{Backups:} PPO is not able to back up to Azure. \cite{PerconaDocuCompare} Although it uses Patroni, which has this ability.
    \item \textbf{Disaster Recovery:} PPO documentation does not mention the possibility of restoring a single database from a backup. \cite{PerconaDocuBackups}
    \item \textbf{Monitoring:} PPO is not using the usual monitoring stack consisting of Prometheus and Grafana but their own Percona Monitoring and Management. \cite{PerconaDocuMonitoring}
\end{itemize}

The current stable version of PPO is 1.4.0 was released on 31st March 2023\footnote[4]{Version 2.0.0 has also been released, it is not yet production-ready yet. Therefore will not be tested or evaluated.}. \cite{PerconaDocuV2} PPO is distributed under the Apache License 2.0, an open-source license that allows for both commercial and non-commercial use. With regards to capability, PPO is considered to have the second highest capability level, labeled as Deep Insights. \cite{OperatorHubPercona}

\n{2}{Summary}
\tab{Summary of selected 0perators}{tab:operatorsSummary}{1}{|l|l|l|l|}{
    \hline Operator & Maturity level & Current production version & Release date\\ \hline
    PGO & Autopilot & 5.3.1 & 17th February 2023\\ \hline
    CNPGO & Autopilot & 1.20.0 & 27th April 2023\\ \hline
    SPGO & Deep Insights & 1.4.3 & 23th February 2022\\ \hline
    PPO & Deep Insights & 1.4.0 & 31st March 2023\\ \hline
}

\n{2}{Key differences}
\tab{Key differences between selected Operators}{tab:operatorsDiffences}{1}{|l|l|l|l|l|}{
    \hline Feature & PGO & CNPGO & SPGO & PPO\\ \hline
    In place Point in time recovery & Yes & No & No & No\\ \hline
    Individual database restore & Yes & No & No & No\\ \hline
    Operator user interface & No & No & Yes & No\\ \hline
    Major version upgrade & Yes & No & Yes & Yes\\ \hline
    Supported Postgres versions & v11 - v15 & v11 - v15 & v12 - v15 & v12 - v14\\ \hline
}

\n{1}{Metrics}
\label{chap:metrics}
This chapter aims to answer the second research question: 'What metrics are suitable for comparing Opereators for lifecycle management in
Kubernetes?'" According to Tom Gilb \cite{gilb1988principles}, the main issue in software attribute requirements is identified not in their functionality, but in their quality. Gilb differentiates these attribute requirements into two categories: Resources (people, time, money), which are always finite, and qualities or benefits, which are always fewer than desired. Knowledge about the functionality that an Operator must provide to achieve a certain level of capabilities is obtained from Chapter~\ref{chap:operators}. The most significant functional properties of Operators have been detailed in Chapter~\ref{chap:searchForOperators}. With the lifecycle of Postgres and the capabilities of operators now understood, what remains to be examined are their qualitative properties. The upcoming testing will be focused on the proposed qualitative metrics.

\n{2}{Performance}
Performance is a qualitative parameter of a system, defined by the efficiency with which the system utilizes allocated resources. In the case of Postgres, performance can be expressed as the number of transactions executed per unit of time. A higher transaction rate is indicative of superior performance.
\n{2}{Reliability}
Reliability is a qualitative parameter that determines the degree of system dependability. If a system cannot be relied upon, it cannot be utilized effectively. Therefore, reliability is a critical parameter of any system.
\n{3}{Maturity}
Maturity in the context of software refers to the degree to which a system is fully developed, perfected and reliable. A mature system is typically characterized by stability, reliability, robustness, and well-defined, predictable behavior based on its previous operation and iterations.
\n{2}{Usability}
Usability is a key aspect of software design that focuses on user experience. It refers to the ability of a system to achieve certain goals for certain users in a particular context of use.
\n{3}{Learnability}
How easy it is for users to learn how to use the system.
\n{3}{Operabilitiy}
How easy is it for users to successfully use the system.
\n{2}{Maintenance}
Activities that are performed after the software is deployed to ensure its correct functionality and performance. Maintenance may include bug fixes, adding new features, performance optimization, updates for compatibility with new systems, etc. Ignoring software maintenance can lead to increased repair costs and reduced system performance over time. Additionally, it can cause system instability, increased vulnerability to security threats, and eventually, potential system failure.
\n{3}{Renewal}
Renewal is considered a crucial aspect of software maintenance, playing a significant role in maintaining the relevance of a software system over time. Essentially, this process ensures that the software is consistently updated to align with current standards and technologies.
\n{2}{Security}
Security refers to the measures, practices, and technologies employed to protect the system and its data from threats and attacks.
\n{3}{Vulnerabilities}
Vulnerability refers to a weakness in a system that can be exploited by a threat actor, such as a hacker, to perform unauthorized actions within a computer system. This can involve gaining access to the system's features and data, or disrupting the system's normal functioning.


\n{1}{Testing methodology}
This chapter aims to answer the third research question: 'What approach should be taken to determine the degree to which the metrics are
met?'" and presents a high-level overview of the testing methodology. The goal of the methodology is to deliver rules and guidance for test process that produces test reports forming the basis of this evaluation.

\n{2}{Notice}
It is important to notice at the beginning of this chapter that testing as described in \cite{FoundationOfSoftwareTesting} has following seven testing princliples:
\begin{enumerate}
    \item Testing shows the presence of defects, not their absence
    \item Exhausting testing is impossible
    \item Early testing saves time and money
    \item Defects clusters together
    \item Beware of pesticide paradox
    \item Testing is context dependent
    \item Absence of errors is fallacy
\end{enumerate}



Therefore, the test process derived from this methodology as every test process will not exhaustively test the Operators and will be depended on thesis context, author bias, and author skills. Because The objective of this thesis is to conduct an extensive evaluation of various Kubernetes
Operators available for Postgres lifecycle management the main scope of this methodology is to deliver test process that will produce test reports that will form the base for this evaluation.

\n{2}{Criteria}
The following criteria described in Chapter \ref{chap:metrics} will be subjected to testing. These criteria should provide a solid basis for the decision-making process. To keep track of these criteria, each one has been assigned an ID. The list of identified criteria is as follows:

\begin{itemize}
    \item	CP: Performance
    \item	CR: Reliability
          \begin{itemize}
              \item	CR1: Maturity
          \end{itemize}
    \item	CU: Usability
          \begin{itemize}
              \item	CU1: Learnability
              \item  CU2: Operabilitiy
          \end{itemize}
    \item	CM: Maintenance
          \begin{itemize}
              \item	CM1: Renewal
          \end{itemize}
    \item	CS: Security
\end{itemize}


\n{2}{Test management process}
According to the IEEE Standard for Software Test Documentation \cite{ieeeTestProcess}, test management processes have three main test processes: test strategy and planning, test monitoring and control, and test completion.
As depicted in Figure \ref{fig:ieeeTestProcess} testing has more than one management test process. The main process is the Organizational process which is further divided into Test management processes that are then devided into Dynamic test processes.

\obr{Test management process relationships \cite{ieeeTestProcess}}{fig:ieeeTestProcess}{1}{graphics/test_management_process.png}

Thesis test process will consist of one management process that will create two managed subprocesses, one for static testing and one for dynamic testing. This test management process will monitor and control subprocesses. Subprocesses will deliver all their deliverables to this main process.

The main management consist of following tasks:
\begin{itemize}
    \item High-level test strategy
    \item High-level test planning
\end{itemize}

Subprocess consist of following tasks:
\begin{itemize}
    \item Test strategy
    \item Test planning
\end{itemize}

\n{2}{Test strategy and planning}
The output of the test strategy and planing will be the test plan, as the basis for its creation will be the requirements created earlier. Details about activities are described by ISO/IEC/IEE 29119-3.

As depicted on \ref{fig:ieeeTestProcess} test plan is not static but it changes according to monitoring.


\n{2}{Test plan}
\obr{Test plan creation activities \cite{ieeeTestProcess}}{fig:ieeeTestPlan}{1}{graphics/test_plan_process.png}
In order to create a test plan, IEEE proposes the following procedure shown in the figure \ref{fig:ieeeTestPlan} with the idea that some activities can be repeated.

The result of a properly designed test plan should be:
\begin{itemize}
    \item Scope of testing
    \item List of identified risks
    \item Testing strategy
    \item Test environment
    \item Test tools
    \item Test data
    \item Staffing
    \item Scheduling
    \item Required training
    \item Estimates of time and resources
    \item Compliance with all stakeholders
\end{itemize}

\n{2}{Test monitoring and control process}
The role of the test monitoring and control process is to observe the test process and detect deviations from the plan. This process controls the test process throughout its duration. The findings are then used to modify the test plan.

To avoid unnecessary bureaucracy, where the manager and tester are the same person, and consequently the testing progress would be reported to the person who is also filing it, there will be no status reports during this process.

\n{2}{Test completion process}
The test completion process as depicted in the figure \ref{fig:ieeeTestCompletition} will be used in testing after each test the test competition report will be created and delivered to a higher level.
\obr{Test completion process \cite{ieeeTestProcess}}{fig:ieeeTestCompletition}{1}{graphics/test_completation_report.png}


\n{2}{Dynamic and static test processes}
According to ISTQB \cite{FoundationOfSoftwareTesting} there are two types of tests. Static and dynamic. The main difference is that the static technique does not execute the tested software, but the dynamic does. The testing process will utilize both techniques using the dynamic testing process depicted in figure \ref{fig:ieeeTestDynamicTestProcess}.

\obr{Dynamic test processes \cite{ieeeTestProcess}}{fig:ieeeTestDynamicTestProcess}{1}{graphics/test_dynamic_test_processes.png}

\n{2}{Test design and implementation processes}
Test design and implementation process must follow the process depicted in the figure \ref{fig:ieeeTestDesignAndImplementation}. Test design techniques should be used to derive test cases. Test cases must be traceable to requirements and must meet the ISO/IEC/IEE 29119-3 requirements. This process can be reentered multiple times and must meet the completion criteria specified in the test plan.
\obr{Test design and implementation \cite{ieeeTestProcess}}{fig:ieeeTestDesignAndImplementation}{1}{graphics/test_design_and_implementation.png}


\n{2}{Test environment and data management processes}
Based on the test plan all the environments must be established and well mainteained.

\n{2}{Test execution process}
Test execution process depicted in \ref{fig:ieeeTestExecutionProcess} must be followed. After the test execution, the execution log must be delivered. Details about activities are described by ISO/IEC/IEE 29119-3.

\obr{Test execution process \cite{ieeeTestProcess}}{fig:ieeeTestExecutionProcess}{1}{graphics/test_execution_process.png}

\n{2}{Test incident report process}
The process for reporting test incidents, as depicted in Figure \ref{fig:ieeeTestExecutionProcess}, must be followed. It is important to note that the purpose of testing is not to simply find incidents, but rather to test the capabilities of the system. As such, the term "finding" will be used in cases where it is more appropriate.

\n{2}{Level of detail}
Please note that this chapter provides only a high-level overview of the testing methodology. More detailed information can be found in ISO/IEC/IEEE 29119-2\footnote[5]{https://standards.ieee.org/ieee/29119-2/7498/}. If there are any doubts regarding the testing process, this standard should be used as a guide to ensure that all necessary aspects of testing are properly addressed.

\cast{APPLICATION OF THEORY}
\n{1}{Test process}
As stated in the methodology, the main test process with a general test plan was created. The general test plan can be seen in Appendix AI. This process was divided into two subprocesses: one for static testing and one for dynamic testing.

According to the priorities stated in the general test plan, the first test process to be conducted was the static test process.

\obr{Test process}{fig:testingTestProcess}{1}{graphics/test_process-2.png}

\n{2}{Static test process}
The static test process was divided into three separate subprocesses, each with its own testing process.
\n{3}{Reliability and maintenance}
Reliability and maintenance testing consists of two parts: renewal and maturity. Both of these parts form the first static testing process because they share the same test items and therefore dividing them into separate processes doesn't make sense. This process was designed to provide the necessary information for decision-making regarding the maintenance quality and maturity level of the Operators. A detailed test plan and further specifics of this process can be found in Appendix A II.

During the test process, it was observed that PPO does not use the repository to track issues, instead it uses Jira. Another challenge arose when trying to determine the repository creation date and getting the number of commits in repository. Retrieving this data from Gitlab or Github proved to be quite difficult. As a workaround, all the repositories were cloned, and the date of the initial commit was obtained using the command mentioned in Listing \ref{lst:gitLogReverse}.
Likewise, the total number of commits was determined using the command mentioned in Listing \ref{lst:commitsCount}. Due to these modifications, the test plan was subsequently revised.

\begin{lstlisting}[language=bash, caption={Commits count}, label={lst:commitsCount}]
        $ git rev-list --all -count
    \end{lstlisting}

\begin{lstlisting}[language=bash, caption={Reverse git log}, label={lst:gitLogReverse}]
        $ git log --reverse
\end{lstlisting}

The results of this repository analysis are presented in Table \ref{tab:operatorTestRelAndMaint}. As can be seen, PGO and PPO share the same creation date. This is due to the fact that PPO is a fork of PGO, as metioned earlier in Chapter~\ref{chap:ppo}.

\tab{Operator repository analysis}{tab:operatorTestRelAndMaint}{.5}{|l|l|l|l|l|}{
    \hline
    & PGO	& CNPGO	& SPGO & PPO \\ \hline
    Repo creation	&  27th Feb 2017 & 18th Feb 2020	&  29th May 2019	&  27th Feb 2017 \\ \hline
    Test date	&   1st May 2023 & 1st May 2023	&  1st May 2023	&  1st May 2023 \\ \hline
    Stars	& 3258	& 1198	 & 84 &	149 \\ \hline
    Issues	&  1884	& 764	& 1959	& 317 \\ \hline
    Issues fixed	&   1755	& 691 &	1514	& 282 \\ \hline
    Commits	&   5582	& 3362 &	7208	& 4689 \\ \hline
}
\n{3}{Learnability}
The learnability testing process was divided into two distinct aspects: the first one being the training required to operate with the Operators, and the second one being the presence of examples concerning Postgres life cycle events in the Operator's documentation. Following the guidelines of the test plan (available in Appendix A III), a checklist was created. The documentation for each Operator was thoroughly reviewed, resulting in the findings presented in Tables \ref{tab:operatorTestTraining} and \ref{tab:operatorTestDocExamples}.

\tab{Training needed}{tab:operatorTestTraining}{1}{|l|l|l|l|l|}{
    \hline
    & PGO	& CNPGO	& SPGO & PPO \\ \hline
    1st training & Kubectl & Kubectl & Kubectl & Kubectl \\ \hline
    2nd training & Kustomize & Helm	& Helm & Helm\\ \hline
    3rd training & 	& Cnpg	&  & \\ \hline
}

\tab{Documentation examples}{tab:operatorTestDocExamples}{1}{|l|l|l|l|l|}{
    \hline
    & PGO	& CNPGO	& SPGO & PPO \\ \hline
    Cluster creation &	Yes &	Yes &	Yes &	Yes \\ \hline
    Minor upgrade &	Yes &	Yes &	No &	Yes \\ \hline
    Major upgrade &	Yes &	No &	No &	Yes \\ \hline
    Backup &	Yes &	Yes &	Yes &	Yes \\ \hline
    Restore &	Yes &	Yes &	Yes &	Yes \\ \hline
    Monitoring &	Yes &	Yes &	Yes &	Yes \\ \hline
    Vertical scaling &	Yes &	Yes &	Yes &	No \\ \hline
    Horizontal scaling &	Yes &	Yes &	Yes &	Yes \\ \hline
    Configuration Update &	Yes &	Yes &	Yes &	Yes \\ \hline
    Uninstall&	Yes &	Yes &	Yes &	No \\ \hline
}

\n{3}{Security}
To proceed with the security testing, a vulnerability analysis test plan was created (which can be found in Appendix A IV), and the process was carried out according to this plan.

According to test plan, the test items for this process were identified, the test tool was downloaded, and a test procedure was developed which consisted of four test cases - one for each operator (details in Appendix A IV). Each operator was then tested according to this procedure, with the overal vulnerability results presented in Table \ref{tab:operatorTestVulnTrivy}.

Trivy was unable to detect any vulnerabilities in CNPG's container image utilizing Debian 11.6. The rest of the operators are using the Red Hat 8.7 container image, which resulted in almost identical vulnerability scores.
Unfortunately, the implementation of Red Hat 8.7 in both PGO and SPGO has a high vulnerability with openssl-libs (CVE-2023-0286).
More details about this vulnerability can be found here: \url{https://avd.aquasec.com/nvd/2023/cve-2023-0286/}.

The absence of any detected vulnerabilities in CNPGO by Trivy raised a question: Is Trivy accurately scanning this image? To address this concern, the test plan was updated to include an additional test case. This test involved scanning the base image of CNPGO to ascertain if Trivy could detect any vulnerabilities in Debian 11.6.
Debian results can be interpreted in two ways. The first interpretation suggests that CNPG might be using Debian but has effectively removed or mitigated the vulnerable parts. The second interpretation considers the possibility that Trivy may not be able to accurately identify Debian vulnerabilities within CNPGO.

To eliminate the second interpretation an additional vulnerability analysis tool, Snyk, was incorporated into the testing process. The test plan was subsequently adjusted, and the images were scanned using Snyk. This alternative method produced results similar to those from the Trivy scanning for each Operator. However, there were exceptions in terms of High severity issues; Snyk identified three additional vulnerabilities in the SSL libraries of PGO and SPGO and provided different results for Debian.

Complete results of Trivy and Snyk scans can be located in the thesis repository\footnote[6]{https://github.com/Ovec/Bachelors-thesis} folder at \url{tests/vulnerability_analysis}. The overall results are presented in Table \ref{tab:operatorTestVulnTrivy} and \ref{tab:operatorTestVulnSnyk}

\tab{Trivy vulnerability analysis results}{tab:operatorTestVulnTrivy}{1}{|l|l|l|l|l|l|}{
    \hline
    & PGO	& CNPGO	& SPGO & PPO & Debian 11.6 \\ \hline
    Critical &	0	& 0 &	0 &	0 &	1 \\ \hline
    High &	1 &	0 &	1 &	0&	17 \\ \hline
    Medium &	40 &	0 &	41 &	35&	6  \\ \hline
    Low &	36 &	0 &	36 &	36&	59  \\ \hline
    Unkown &	0 &	0 &	0 &	0 &	0 \\ \hline
}

\tab{Snyk vulnerability analysis results}{tab:operatorTestVulnSnyk}{1}{|l|l|l|l|l|l|}{
    \hline
    & PGO	& CNPGO	& SPGO & PPO & Debian 11.6 \\ \hline
    Critical & 0 & 0 &	0 &	0 &	0 \\ \hline
    High & 4 &	0 &	4 &	0 &	1 \\ \hline
    Medium &38 &	0 &	39 &	36 &	2  \\ \hline
    Low &	39 &	0 &	39 &	39 &	48 \\ \hline
    Unkown &	0 &	0 &	0 &	0 &	0 \\ \hline
}

\n{2}{Dynamic test process}

The dynamic test process was divided into two separate subprocesses, each with its own
testing process.

\n{3}{Environments}

\n{3}{Usability}
\n{3}{Performance}

\n{1}{Evaluation}


\n{2}{Reliability and maintenance}

In order to determine the maturity of the system, data was statically collected from the repositories of each operator. This included the popularity of the operators, the number of issues they had, and the number of issues that were resolved. The maturity was then determined based on the popularity of the operators and the ratio of resolved issues.

\n{3}{Popularity}
Popularity can be considered an indicator of maturity because broad recognition and use of a product often suggest that the product is reliable, efficient, and capable of meeting user requirements. In the context of software, popular software tends to be more thoroughly tested, leading to the identification and correction of bugs, and therefore greater stability and maturity.

The operator with the highest number of stars, and therefore the most popular, is the PGO operator. To compare operators, the most popular one was assigned a popularity score of 100\%, and the others were assigned a score relative to their share of the maximum.

\tab{Popularity of Operators}{tab:operatorMaturity}{1}{|l|l|l|l|l|}{
    \hline
    & PGO	& CNPGO	& SPGO & PPO \\ \hline
    Popularity (stars)	& 3258	& 1198	 & 84 &	149 \\ \hline  \hline
    Popularity ratio	& 100\% &	37\%	& 3\% &	5\% \\ \hline
}

\n{3}{Fixed issues}
The number of issues resolved is often considered a sign of a software project's maturity, as it indicates that the developers are actively maintaining the software, responding to user feedback, and fixing problems as they arise. A high number of resolved issues indicates that the project has faced and overcome challenges, which may indicate that the software has been "battle tested" and improved over time.

The operator with the highest proportion of resolved issues is PGO, which has successfully addressed 93\% of its reported issues. CNPGO and PPO exhibit a reasonable performance level in this area. However, SPGO, despite being less popular and more recent than PGO, has more reported issues and a large number of unresolved issues (445 in total). This could suggest that SPGO is still underdeveloped.

\tab{Operators issues}{tab:operatorIssues}{1}{|l|l|l|l|l|}{
    \hline
    & PGO	& CNPGO	& SPGO & PPO \\ \hline
    Total issues	&  1884	& 764	& 1959	& 317 \\ \hline
    Issues fixed	&   1755	& 691 &	1514	& 282 \\ \hline  \hline
    Fixed issues ratio	&  93\% &	90\% &	77\% &	89\% \\ \hline
}

The overall maturity of the operators was calculated as an average of popularity and the ratio of fixed issues. The results can be seen in Table \ref{tab:operatorMaturity}. As maturity is the only subtest under reliability, it is considered as a measure of reliability.

\tab{Operators maturity}{tab:operatorMaturity}{1}{|l|l|l|l|l|}{
    \hline
    & PGO	& CNPGO	& SPGO & PPO \\ \hline
    Overal maturity / reliability	& 97\%	& 64\% &	40\%	& 47\% \\ \hline
}

\n{3}{Renewal}
Software renewal, a key aspect of software maintenance, is often essential to ensure that the software continues to be effective and efficient in light of changing user needs, technological progress, and evolving industry standards. Commits to the repository represent incremental changes in the project. Therefore, the number of commits per unit of time indicates the rate at which the software is being renewed.

The overall renewal rate of the operators depicted in Table \ref{tab:operatorRenewal} was calculated as the number of commits since the day of repository creation. The highest ratio of commits per day was achieved by SPGO with an average of 5.03 commits per day, earning it a score of 100\%, while the other operators were assigned scores relative to their share of this maximum.

Interestingly, this contrasts with the Operator issues where SPGO scored the lowest and was therefore considered underdeveloped. A possible explanation for this could be that SPGO uses issues for its own project management, rather than solely for issue tracking. As renewal is the only subtest under maintenance, it is considered as a measure of maintenance.


\tab{Operators renewal}{tab:operatorRenewal}{1}{|l|l|l|l|l|}{
    \hline
    & PGO	& CNPGO	& SPGO & PPO \\ \hline
    Repo creation	&  27th Feb 2017 & 18th Feb 2020	&  29th May 2019	&  27th Feb 2017 \\ \hline
    Test date	&   1st May 2023 & 1st May 2023	&  1st May 2023	&  1st May 2023 \\ \hline
    Sum of commits	&   5582	& 3362 &	7208	& 4689 \\ \hline
    Commits/day	&   2.48	& 2.88	& 5.03	& 2.08 \\ \hline  \hline
    Renewal / maintenance &   49\%	&  57\%	&  100\%	& 41\% \\ \hline
}

\n{2}{Learnability}
Learnability testing, which is one part of usability, aims to assess the ease with which new users can understand and use a product. This particular testing was conducted with two primary objectives. The first objective was to identify the presence of examples in the documentation, as learning facilitated by examples is generally more effective than without. The second objective was to gauge the level of learning required to operate the system effectively.


\n{3}{Required training}
The training required presented in Table \ref{tab:operatorTraining} was calculated as the additional training needed to work with each operator, over and above knowledge of Kubectl (the Kubernetes command-line tool).
Each additional tool required to work successfully with the operator resulted in a 5\% score decrease.

\textcolor{red}{TBD - this must be in testing. Learnability testing table must be update in dynamic testing becuase it was relized that the cnpg plugin is not necessary for cnpg to operate.}
% In this regard, CNPGO was at a disadvantage because, according to its documentation, the Cnpg plugin is necessary to uninstall it. PGO also faced a similar issue as it utilizes the simple Kustomize tool to customize its manifest with a namespace.

\tab{Training needed}{tab:operatorTraining}{1}{|l|l|l|l|l|}{
    \hline
    & PGO	& CNPGO	& SPGO & PPO \\ \hline
    Required Training &  95\%	& 95\% &	95\% &	95\%\\ \hline
}

\n{3}{Documentation examples}
As presented in Table \ref{tab:operatorTestDocExamples}, documentation examples were prevalent across all operators. SPGO, however, is at a disadvantage here due to its own GUI (Graphical User Interface), which is capable of managing the entire cluster and can easily handle all cluster operations. Surprisingly, it does not provide examples for major and minor updates, which can be easily configured via the GUI, and it would also be beneficial if the remaining examples provided some indication that the respective functionality could be conveniently implemented using the GUI. Considering that there are ten examples, each example was assigned a value of 10\% towards the total score.

\tab{Documentation examples}{tab:operatorDocExamples}{1}{|l|l|l|l|l|}{
    \hline
    & PGO	& CNPGO	& SPGO & PPO \\ \hline
    Documentation examples &	100\%	& 90\%	& 80\%	& 80\% \\ \hline
}


The overall learnability rating of the Operators, as depicted in Table \ref{tab:operatorLearnability}, was calculated as the average of the scores from the required training and the documentation examples.

Overall, the learnability scores are quite high, suggesting that all of the operators are relatively easy to learn. If monitoring is not a requirement, knowledge of Kubectl commands is sufficient to work with the CNPGO, SPGO, and PPO Operators. However, to install monitoring, Helm is necessary for these operators. For PGO, learning Kustomize for cluster operations is sufficient, and no additional tool is needed for monitoring.

\tab{Learnability of Operators}{tab:operatorLearnability}{1}{|l|l|l|l|l|}{
    \hline
    & PGO	& CNPGO	& SPGO & PPO \\ \hline
    Learnability	& 98\% &	93\% &	88\% &	88\%\\ \hline
}

\n{2}{Security}
The results of vulnerability analysis presented in Table \ref{tab:operatorTestVulnTrivy} and \ref{tab:operatorTestVulnSnyk} were quantified in the following manner: the presence of a critical vulnerability was scored as 0\%, a high severity vulnerability as 20\%, medium as 40\%, low as 60\%, unknown as 80\%, and no vulnerabilities as 100\%. CNPGO was the only operator with zero detected vulnerabilities, thus achieving a full score and can be considered secure. PPO, having only medium and lower severity vulnerabilities, can also be considered reasonably secure. However, both PGO and SPGO, which had high-severity vulnerabilities, can be seen as less secure compared to CNPGO and PPO.

\tab{Vulnerability analysis}{tab:operatorVuln}{1}{|l|l|l|l|l|}{
    \hline
    & PGO	& CNPGO	& SPGO & PPO \\ \hline
    Security &	20\% &	100\%	& 20\%	& 40\% \\ \hline
}
\n{2}{Operability}
\label{chap:testOperatbility}
To evaluate the ease of working with each Operator, use cases derived from the Postgres lifecycle (as described in chapter \ref{chap:lifecycle}) were created. A separate test case was developed for each use case, and the Operators were evaluated accordingly.
Testing was conducted on a local Kind Kubernetes cluster set up on a first generation M1 Macbook Air with 8GB of RAM and a 500GB disk size.

For specific test cases, such as cluster monitoring for SPGO and PPO, and cluster restore for CNPGO, SPGO, and PPO, the Operators could not proceed with the local Kind cluster. Therefore, Google Kubernetes Engine (GKE) cluster was utilized. The configurations for both GKE and Kind clusters can be located in the repository folder at \url{tests/environments}.
Use cases, their scenarios, and the test plan can be found in Appendix A V.

Operability depicted in Table \ref{tab:operatorOperability} was evaluated based on two aspects. The first aspect, 'ease of use,' evaluated the number of command executions needed to achieve each goal. The second aspect focused on the quality of the Operator's monitoring capabilities.

\tab{Operability}{tab:operatorOperability}{1}{|l|l|l|l|l|}{
    \hline
    & PGO	& CNPGO	& SPGO & PPO \\ \hline
    Ease of use &	32\% & 25\% &	100\% &	17\% \\ \hline
    Monitoring &	83\% & 75\% &	50\% &	75\% \\ \hline  \hline
    Operability &	58\% &	50\%	& 75\%	& 46\% \\ \hline
}

\n{3}{Ease of use}
The number of commands executed to achieve the goal was counted for each test case. If an Operator did not provide the necessary functionality, the number of steps required was designated as 10. This is because realizing this functionality would require significant effort, or it might not be achievable at all.
Test procedures can be found in the repository folder at \url{tests/operability/ease_of_use}.

The operator with the fewest steps was assigned a score of 100\%. The scores for the other operators were calculated relative to this maximum. The results are depicted in Table \ref{tab:operatorEaseOfUse}.

\n{4}{SPGO}
The reason SPGO scored the highest is due to its user-friendly interface. Most tasks can be accomplished directly within this interface, eliminating the need for using the terminal. Even complex tasks, such as performing a major version upgrade, can be easily carried out via this user interface. An example of this user interface is shown in Figure \ref{fig:spgoUI7}. Additional images can be found in the repository folder at \url{doc/graphics/monitoring/SPGO}.

\n{4}{Cluster major version upgrade}
The upgrade to a new major version appears to be the most challenging task for each Operator. While SPGO handled the cluster upgrade seamlessly, PGO required four steps to proceed with the major version upgrade. On the other hand, CNPGO claimed to be capable of an "Offline import of existing PostgreSQL databases, including major upgrades of PostgreSQL". This process involves dumping the database and restoring it to a new cluster, which can be done with any cluster and is not considered a major upgrade. PPO declared in their documentation that they are capable of automatic updates even between versions. However, this Jira issue (\url{https://jira.percona.com/projects/K8SPG/issues/K8SPG-254?filter=allopenissues}) suggests otherwise.

\n{3}{Monitoring}
To evaluate the monitoring capabilities of each operator, a set of necessary monitoring parameters was created. During the monitoring deployment test case, screenshots of each monitoring system were taken. These screenshots can be found in the repository folder at \url{doc/graphics/monitoring}. Each operator's score was determined based on the number of parameters covered by its monitoring system.

All of the operators, except for PPO, use the traditional Grafana Prometheus monitoring stack, while PPO uses the Percona Monitoring and Management solution. This Percona monitoring system is quite extensive, but its coverage of parameters is comparable to that of the less extensive CNPGO monitoring system.

It can be derived that all operators, with the exception of SPGO, offer high-quality monitoring. While SPGO is easy to operate, it is difficult to monitor effectively.

\obr{SPGO's user interface}{fig:spgoUI7}{1}{graphics/monitoring/SPGO/spgo_ui7.png}

\tab{Ease of use}{tab:operatorEaseOfUse}{.5}{|l|l|l|l|l|}{
    \hline
    & PGO	& CNPGO	& SPGO & PPO \\ \hline
    Operator installation & 2 & 1 & 2 & 3 \\ \hline
    Cluster installation & 1 &	2 &	0 &	1\\ \hline
    Cluster monitoring & 2 & 4 & 3 & 5 \\ \hline
    Cluster vertical scaling & 1 &	1 &	0 &	1 \\ \hline
    Cluster horizontal scaling & 1 & 1 & 0 & 1\\ \hline
    Cluster connection pooling & 1 & 1 & 0 & 0 \\ \hline
    Cluster extension install & 1	& 1 & 0 & 10 \\ \hline
    Cluster number of connections increase & 1 &	1 &	0 &	2 \\ \hline
    Cluster max\_wall\_size increase & 1 & 1 & 0 & 3 \\ \hline
    Cluster scheduled backup & 1 & 1 & 0 & 1 \\ \hline
    Cluster ad-hoc backup & 2	& 1 & 0 & 1 \\ \hline
    Cluster restore & 2 & 1 & 1 & 1 \\ \hline
    Cluster minor version upgrade & 1 &	1 &	0 &	1 \\ \hline
    Cluster major version upgrade & 4 &	10 & 0 & 10 \\ \hline
    Operator uninstall & 1	& 1 & 1 & 1 \\ \hline
    Cluster uninstall  & 1 & 1 & 0	& 1 \\ \hline
    Overal & 22 & 28 & 7 & 41 \\ \hline  \hline
    Ease of use &	32\% & 25\% &	100\% &	17\% \\ \hline
}

\tab{Monitoring}{tab:operatorMonitoring}{.5}{|l|l|l|l|l|}{
    \hline
    & PGO	& CNPGO	& SPGO & PPO \\ \hline
    Health & Yes &	Yes &	Yes &	Yes\\ \hline
    Query performance & Yes	& Yes	& No	& Yes\\ \hline
    Number of connections & Yes	& Yes	& Yes	 &Yes \\ \hline
    Locks & Yes	& Yes	& Yes	& Yes \\ \hline
    Index hit  & No	& No	& No	& No\\ \hline
    Cache hit & Yes	& No	& Yes	& Yes \\ \hline
    Disk space usage & Yes	& Yes	& No	& Yes \\ \hline
    CPU and memory usage & Yes	& Yes	& Yes	& Yes \\ \hline
    WAL generation rate & Yes	& Yes & No	& Yes \\ \hline
    Replication lag & Yes	& Yes	& No	& No \\ \hline
    Errors and logs & No	& No	& No	& Yes \\ \hline
    Backup and recovery & Yes	& Yes	& Yes	& No \\ \hline  \hline
    Monitoring &	83\% & 75\% &	50\% &	75\% \\ \hline
}

\n{2}{Performance}

\obr{Kubernetes performance setup}{fig:testingGKEperformance}{1}{graphics/performance_setup.png}

In order to perform performance testing, it was necessary to create a more robust cluster than the Kind cluster on MacBook Air. This was achieved by creating a Google Kubernetes Engine cluster with three e2-standard-2 nodes. Each node was equipped with 2 virtual CPUs and 8 GB of RAM. A standalone Postgres node with pgbench was also deployed to the cluster. A simplified version of this cluster is depicted in Figure \ref{fig:testingGKEperformance}. The Terraform plans for this deployment can be found in the \url{tests/environments/gke} repository directory.

For each Operator test, the Kubernetes cluster was recreated to ensure a clean starting point. A Postgres cluster with three nodes (one primary and two replicas) was deployed, along with three connection pooler instances.

Cluster configurations were also adjusted according to the recommendations from the pgTune website. The adjusted settings were as follows:
\begin{itemize}
    \item max\_connections: 200
    \item shared\_buffers: 1536MB
    \item effective\_cache\_size: 4608MB
    \item maintenance\_work\_mem: 384MB
    \item checkpoint\_completion\_target: 0.9
    \item wal\_buffers: 16MB
    \item default\_statistics\_target: 100
    \item random\_page\_cost: 4
    \item effective\_io\_concurrency: 2
    \item work\_mem: 3932kB
    \item min\_wal\_size: 1GB
    \item max\_wal\_size: 4GB
\end{itemize}

This configuration was implemented to standardize each cluster configuration.

After the deployment of the Postgres cluster, each Operator was tested using the pgBench tool, a PostgreSQL benchmarking utility. This tool was configured to execute 10,000 transactions across 25 concurrent clients and utilizing 10 threads. This benchmark was directed towards the Postgres cluster's pooler service. This procedure was replicated twice more for thoroughness.

The results depicted in Table \ref{tab:operatorPerformance} only display the transactions per second, as these offer a sufficient indication of the operators' performance. For a comprehensive view of the results from this benchmark, along with the Test Completion Report, please refer to Appendix A VI.

The commands executed throughout this procedure, along with the complete configuration of the operators, can be located in the \url{tests/performance} directory in the repository.

\n{3}{Issues with CNPGO}
CNPGO was the only one unable to execute 250,000 transactions in each run due to an error (client 6 script 0 aborted in command 4 query 0: FATAL: query wait timeout, SSL connection has been closed unexpectedly).
This error usually occurs when a query takes too long to execute, leading to a timeout. The SSL connection is then closed unexpectedly, causing the transaction to fail. This might suggest that CNPGO is struggling with performance or network stability in this particular scenario.

\n{3}{Issues with SPGO}
The possible reason for SPGO's low transactions per second score is that a cluster profile is needed to deploy an SPGO cluster. When this profile was correctly set, Google Kubernetes Engine was unable to deploy the cluster. By gradually reducing these values, the available resources were eventually found, but these settings were probably too low for optimal performance (500m CPU and 2Gi RAM). Despite the cluster showing that it had more memory and CPU allocable, as may be seen in Figure \ref{fig:testingGKEnodes}, the reduced resource allocation might have constrained SPGO's performance.

\n{3}{Issues with PPO}
As described in Chapter~\ref{chap:testOperatbility}, changes to PPO's configuration do not affect the cluster, therefore, the cluster was not modified during this test. This means that the performance results for PPO are based on its default configuration settings.


\tab{Performance analysis}{tab:operatorPerformance}{1}{|l|l|l|l|l|}{
    \hline
    & PGO	& CNPGO	& SPGO & PPO \\ \hline
    First run &	544.91 tps	& 403.70 tps	& 284.39 tps &	401.46 tps \\ \hline
    Second run &	543.29 tps &	402.54 tps &	279.89 tps &	392.04 tps\\ \hline
    Third run &	538.51 tps &	392.63 tps &	309.16 tps &	387.79 tps \\ \hline
    Mean &	542.24 tps &	399.62 tps &	291.15 tps &	393.77 tps \\ \hline  \hline
    Performance &	100\%	& 74\%	& 54\%	& 73\% \\ \hline
}

The most performant Operator PGO received a score of 100\% in this test, while the other operators were assigned scores proportionally based on their performance.

\obr{GKE nodes details}{fig:testingGKEnodes}{1}{graphics/gke_nodes.png}


\n{2}{Usability}
The overall usability was calculated as the average of learnability, as shown in Table \ref{tab:operatorTraining}, and operability, as presented in Table \ref{tab:operatorOperability}. The results are shown in Table \ref{tab:operatorUsability}.

\tab{Usability}{tab:operatorUsability}{1}{|l|l|l|l|l|}{
    \hline
    & PGO	& CNPGO	& SPGO & PPO \\ \hline
    Learnability & 98\% &	93\% &	88\% &	88\% \\ \hline
    Operability &	58\% & 50\% & 75\% & 46\%\\ \hline \hline
    Usability &	78\%	& 72\%	& 82\%	& 67\% \\ \hline
}

\n{1}{Evaluation}

\tab{Overall quality of Operators}{tab:operatorOveralQuality}{1}{|l|l|l|l|l|}{
    \hline
    & PGO	& CNPGO	& SPGO & PPO \\ \hline
    Performance &	100\% & 74\% & 54\% & 73\% \\ \hline
    Reliability &	97\% & 64\% & 40\% & 47\%\\ \hline
    Usability &	78\% &	72\% &	82\% &	67\% \\ \hline
    Maintenance &	49\% & 57\% & 100\% & 41\% \\ \hline
    Security &	20\% & 100\% & 20\% & 40\% \\ \hline \hline
    Quality &	68.8\% & 73.4\%	& 59.2\% & 53.6\% \\ \hline
}


% ============================================================================ %
\nn{Závěr}


Text závěru.




% ============================================================================ %
