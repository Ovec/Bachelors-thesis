% ============================================================================ %
% Encoding: UTF-8 (žluťoučký kůň úpěl ďábelšké ódy)
% ============================================================================ %

% ============================================================================ %
\nn{Introduction}
TBD - První řádek prvního odstavce v kapitole či podkapitole se neodsazuje, ostatní ano. Vertikální odsazení mezy odstavci je typycké pro anglickou sazbu; czech babel toto respektuje, netřeba do textu přidávat jakékoliv explicitní formátování, viz ukázka sazby tohoto textu s následujícím odstavcem).

Formátování druhého odstavce. Text text text text text text text text text text text text.


% ============================================================================ %
\cast{Theory}

\n{1}{Postgres}
PostgreSQL is an object-relational database management system (ORDBMS) based on POSTGRES, Version 4.2, developed at the University of California at Berkeley Computer Science Department. POSTGRES pioneered many concepts that only became available in some commercial database systems much later. \cite{HRW1997}

PostgreSQL is a powerful, open-source database management system with an enviable reputation for high performance and stability. [2]


\n{1}{Nadpisy a podnadpisy}
Na této stránce je k vidění způsob tvorby různých úrovní nadpisů.

\n{2}{Podnadpis A}
Text

\n{2}{Podnadpis B}
Text

\n{2}{Podnadpis C}
Text

\n{3}{Podpodnadpis alfa}
Text

\n{3}{Podpodnadpis beta}
Text

\n{3}{Podpodnadpis gama}
Text

\n{2}{Podnadpis D}
Text


\n{1}{Vkládání obrázků, tabulek a citací}
Níže následují ukázky vložení obrázku, tabulky a různorodých citací.


\n{2}{Obrázek}
Obrázek \ref{fig:logo} prezentuje logo Fakulty aplikované informatiky.

% Obrázek lze vkládat pomocí následujícího zjednodušeného stylu, nebo klasickým LaTex způsobem
% Pozor! Obrázek nesmí obsahovat alfa kanál (průhlednost). Jde to totiž proti požadovanému standardu PDF/A.
\obr{Popisek obrázku}{fig:logo}{0.5}{graphics/logo/fai_logo_cz.png}


\n{2}{Tabulka}
Tabulka \ref{tab:priklad} obsahuje dva řádky a celkem 7 sloupců.

% Tabulku lze vkládat pomocí následujícího zjednodušeného stylu, nebo klasickým LaTex způsobem
\tab{Popisek tabulky}{tab:priklad}{0.65}{|l|c|c|c|c|c|r|}{
  \hline
   & 1 & 2 & 3 & 4 & 5 & Cena [Kč] \\ \hline
  \emph{F} & (jedna) & (dva) & (tři) & (čtyři) & (pět) & 300 \\ \hline
}


\n{2}{Citování}
Následuje ukázka odkazování na různé zdroje:
\begin{itemize}
	\item kniha \cite{HRW1997},
	\item kapitola v knize \cite{Delorme2006},
	\item článek v odborném žurnálu \cite{Bourreau2006},
	\item konferenční příspěvek \cite{Judish1999},
	\item doktorská práce \cite{Valente2005},
	\item technická zpráva \cite{Fralick1997},
	\item webová stránka \cite{WWWCST}.
\end{itemize}


% ============================================================================ %

% Pokud Vaše práce obsahuje analytickou část, stačí odkomentovat nasledujících dva řádky
%\cast{Analytická část}
%\n{1}{Nadpis}


% ============================================================================ %
\cast{Praktická část}
\n{1}{Nadpis první kapitoly praktické části}
Text


% ============================================================================ %
\nn{Závěr}
Text závěru.


% ============================================================================ %
