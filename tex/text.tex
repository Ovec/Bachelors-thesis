% ============================================================================ %
% Encoding: UTF-8 (žluťoučký kůň úpěl ďábelšké ódy)
% ============================================================================ %

% ============================================================================ %
\nn{Introduction}
We live in the age of the cloud. The cloud has made our jobs easier. We no longer have to physically connect new machines to the network, configure network connections, add disks, or even plug in those virtual machines. Kubernetes together with the cloud can automatically allocate new resources for our applications and thanks to operators stateful services such database system
for us. Thanks to Kubernetes Operators, it can also automatically set up scaling or backups. It can even restore the entire database system from backup. This paper focuses on Kubernetes operators for the popular Postgres database management system. Its goal is to find Operators for Postgres. To evaluate their pros and cons. To test them and recommend the best one.

% ============================================================================ %
\cast{Theory}

\n{1}{Background}
This chapter introduces the key technologies used in this thesis including Postgres, Kubernetes, and Kubernetes operators.
\n{2}{Postgres}
PostgreSQL is a powerful object-relational database management system (ORDBMS) derived from the POSTGRES package written at the University of California at Berkeley. \cite{docuPgwhatIsPg} \cite{pg14introduction} The first version of POSTGRES was released in June 1989. POSTGRES has been used in many applications, including financial data analysis systems, asteroid tracking databases, medical information database, and several geographic information systems. The size of external community users has nearly doubled by 1993. \cite{docuPgBriefHistory}

POSTGRES was using its POSTQUEL query language from version one until 1995, when Andrew Yu and Jolly Chen introduced SQL to POSTGRES. The name has changed to Postgres95. Postgres95 was completely ANSI C code reduced by 25 \% and was 30 – 50 \% faster than Postgres 4.2.  \cite{docuPgBriefHistory}

It was clear by 1996 that the name would not stand the test of time therefore it has been renamed to PostgreSQL. As stated by PostgreSQL documentation \cite{docuPgBriefHistory}: “Many people continue to refer to PostgreSQL as “Postgres” (now rarely in all capital letters) because of tradition or because it is easier to pronounce. This usage is widely accepted as a nickname or alias.“ This thesis will use Postgres as an alias for PostgreSQL as well.

More than 30 years after the first version Postgres has been considered the most used ORDBMS for professional developers by Stack Overflow survey \cite{so2022survey}. According to Riggs and Ciolli \cite{pg14introduction}: “The PostgreSQL feature set attracts serious users who have serious applications. Financial services companies may be PostgreSQL's largest user group, although governments, telecommunication companies, and many other segments are strong users as well.“ It is fully ACID compliant \cite{juba2015learningTransactionIsolation} and supports many kinds of data models such as relational, document, and key/value. \cite{pg14introduction}

%Database engines survey https://db-engines.com/en/ranking
%Describe postgres usage
%Describe scaling etc.
%DSK time consuming Postgres

\n{2}{Kubernetes}
Kubernetes, also known as K8s, is an open-source platform for automating deployment, scaling, and management of containerized applications. It provides a way to manage and orchestrate containers, which are units of software that package up an application and its dependencies into a single, isolated package that can run consistently on any infrastructure. \cite{vayghan2019kubernetes}

As described by Kubernetes Documentation \cite{docuKubeComponents} Kubernetes provides several key features, including:
\begin{itemize}
  \item \textbf{Service discovery:} A container can be exposed by Kubernetes either through its DNS name or its own IP address.
  \item \textbf{Load balancing:} In the case of high traffic to a container, stability of the deployment can be ensured by Kubernetes load balancing and distributing the network traffic.
  \item \textbf{Storage Orchestration:} Storage orchestration in Kubernetes allows for the automatic mounting of a storage system of choice, including local storage, public cloud providers, and others.
  \item \textbf{Automated rollouts and rollbacks:} The desired state of deployed containers can be described using Kubernetes, and the actual state can be changed to the desired state at a controlled rate. For instance, the automation of Kubernetes can be utilized to create new containers for the deployment, remove existing containers, and transfer all their resources to the newly created container.
  \item \textbf{Automatic bin packing:} A cluster of nodes for running containerized tasks is provided to Kubernetes. The amount of CPU and memory required by each container is specified to Kubernetes. The optimal utilization of resources can be achieved by Kubernetes fitting the containers onto the nodes.
  \item \textbf{Self healing:} Containers that fail are restarted by Kubernetes, those that do not respond to the user-defined health check are replaced or killed, and they are not advertised to clients until they are deemed ready to serve.
  \item \textbf{Secret and configuration management:} Sensitive information, such as passwords, OAuth tokens, and SSH keys, can be stored and managed by Kubernetes. The deployment and updating of secrets and application configuration can be done without the need to rebuild container images and without the exposure of secrets in the stack configuration.
\end{itemize}
\n{3}{Kubernetes Components}
Kubernetes cluster is composed of a set of worker machines that run containerized applications called nodes. Each cluster must have at least one node. \cite{docuKubeComponents}
\obr{The components of a Kubernetes cluster \cite{docuKubeComponents}}{}{1}{graphics/kubernetes_cluster_components.png}

The Kubernetes control plane is the management system of a Kubernetes cluster, responsible for maintaining the desired state of the cluster. It consists of multiple components that work together to manage the cluster and its resources, including pods, services, and volumes. The key components of control plane are \cite{masteringKubernetesConcepts}:
\begin{itemize}
  \item \textbf{kube-apiserver:} Acts as the front-end for the Kubernetes API and exposes the API to other components. \cite{docuKubeComponents}
  \item \textbf{etcd:} Highly available distributed key-value store that serves as the backing store for the cluster's configuration data. \cite{Dobies2020}
  \item \textbf{kube-scheduler:} Assigns work to nodes in the cluster, such as scheduling pods to run on nodes. \cite{kubeUpAndRunningPods}
  \item \textbf{kube-controller-manager:} Monitors the cluster's state and makes adjustments as necessary to maintain the desired state. \cite{masteringKubernetesConcepts}
  \item \textbf{cloud-controller-manager:} Manages cloud-related tasks such as node creation and management, volume management, and load balancing, allowing the other components of the control plane to focus on their specific responsibilities. Cloud manager is optional. Can be avoided when Kubernetes not used in cloud. \cite{docuKubeComponents}
\end{itemize}
\textbf{Node components:}
Node components in a Kubernetes cluster run on each node and provide crucial functionality for the operation of containers on that node. \cite{docuKubeComponents}
\begin{itemize}
  \item \textbf{kubelet:} Is responsible for communicating with the control plane and ensuring that containers are running and healthy. \cite{kubeUpAndRunning}
  \item \textbf{kube-proxy:} Is responsible for maintaining network rules on the nodes, allowing network communication to the containers. It enables the containers in a pod to communicate with other containers and the outside world, and performs tasks such as load balancing and traffic routing. \cite{kubeUpAndRunning}
  \item \textbf{container runtime:} Is responsible for running containers. \cite{docuKubeComponents}
\end{itemize}

\n{3}{Kubernetes Concepts}
Pod is the smallest deployable unit that can be created in Kubernetes. \cite{docuKubePods} A Pod in Kubernetes is comprised of multiple containers and storage volumes that are run together within the same execution environment. Pods, not individual containers, are considered to be the smallest unit that can be deployed in a cluster. As a result, all containers included in a single Pod will always run on the same machine. \cite{kubeUpAndRunningPods}

A Pod's specifications are outlined in a Pod manifest, which is simply a JSON or YAML text file that represents the Kubernetes API object. Kubernetes follows a declarative configuration approach, where the system's desired state is defined in a configuration file, and the service then implements the necessary changes to make the desired state a reality. \cite{docuKubeStaticPod}

ReplicaSet’s purpose is to ensure a consistent number of replica Pods are running at all times. It is commonly used to guarantee a specified number of identical Pods are available. However, a Deployment is a more advanced concept that oversees ReplicaSets and provides a more streamlined way to make updates to Pods. It also offers additional features. As a result, it's advisable to use Deployments instead of directly utilizing ReplicaSets, unless you have specific update requirements or don't need to make updates at all. \cite{docuKubeReplicaset}

Service is an abstraction layer and defines a group of Pods and the method to access them (often referred to as a micro-service). The group of Pods targeted by a Service is usually specified through a selector. The Service abstraction makes this possible by enabling the decoupling of components. \cite{docuKubeSevice}
\n{2}{Running Postgres in Kubernetes}
\n{2}{Architecture}


%\obr{2022 Developer Survey \cite{so2022survey}}{}{1}{graphics/postgres_stack_overflow_survey.png}





\n{1}{Nadpisy a podnadpisy}
Na této stránce je k vidění způsob tvorby různých úrovní nadpisů.

\n{2}{Podnadpis A}
Text

\n{2}{Podnadpis B}
Text

\n{2}{Podnadpis C}
Text

\n{3}{Podpodnadpis alfa}
Text

\n{3}{Podpodnadpis beta}
Text

\n{3}{Podpodnadpis gama}
Text

\n{2}{Podnadpis D}
Text


\n{1}{Vkládání obrázků, tabulek a citací}
Níže následují ukázky vložení obrázku, tabulky a různorodých citací.


\n{2}{Obrázek}
Obrázek \ref{fig:logo} prezentuje logo Fakulty aplikované informatiky.

% Obrázek lze vkládat pomocí následujícího zjednodušeného stylu, nebo klasickým LaTex způsobem
% Pozor! Obrázek nesmí obsahovat alfa kanál (průhlednost). Jde to totiž proti požadovanému standardu PDF/A.
\obr{Popisek obrázku}{fig:logo}{0.5}{graphics/logo/fai_logo_cz.png}


\n{2}{Tabulka}
Tabulka \ref{tab:priklad} obsahuje dva řádky a celkem 7 sloupců.

% Tabulku lze vkládat pomocí následujícího zjednodušeného stylu, nebo klasickým LaTex způsobem
\tab{Popisek tabulky}{tab:priklad}{0.65}{|l|c|c|c|c|c|r|}{
  \hline
  & 1 & 2 & 3 & 4 & 5 & Cena [Kč] \\ \hline
  \emph{F} & (jedna) & (dva) & (tři) & (čtyři) & (pět) & 300 \\ \hline
}

% ============================================================================ %

% Pokud Vaše práce obsahuje analytickou část, stačí odkomentovat nasledujících dva řádky
%\cast{Analytická část}
%\n{1}{Nadpis}


% ============================================================================ %
\cast{Praktická část}
\n{1}{Nadpis první kapitoly praktické části}
Text je text


% ============================================================================ %
\nn{Závěr}
Text závěru.


% ============================================================================ %
